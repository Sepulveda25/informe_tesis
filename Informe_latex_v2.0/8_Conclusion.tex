\chapter{\Large Conclusión}
Haber desplegado una solución SIEM y en consecuencia, dotar al incipiente CSIRT de la Universidad Nacional de Córdoba con herramientas de monitoreo y gestión de eventos de seguridad de la información, permitió sumar nuevas capacidades de ciberseguridad a la Universidad. Estas consisten en poder monitorear el tráfico desde y hacia la universidad, incluyendo sus dependencias, infraestructura y activos de información.
 \par
Se logró desarrollar el sistema de monitoreo mediante el despliegue y configuración de Security Onion. A partir de este momento la Universidad pudo observar y analizar los eventos de seguridad de la información que estaban ocurriendo en su tráfico de datos.  \par
Fueron enfrentados varios desafíos, algunos de los cuales pudieron superarse con éxito y otros permanecen como tareas pendientes para posteriores trabajos. En cuanto a los problemas que pudieron ser resueltos se destacan la priorización de categorías de eventos, el aprendizaje y uso de las correlaciones que resuelve ElastAlert. Además fue posible aprender sobre la creación de filtros de eventos para disparar notificaciones específicas, identificar el comportamiento de determinados tipos de eventos, la administración inteligente del espacio en el disco y el desarrollo de scripts en TheHive. \par
En cuanto a las oportunidades de mejora identificadas para posteriores trabajos, se encuentran: el despliegue y optimización de nodos de almacenamiento, especificar acciones automáticas según el tipo de ataque así como la identificación de las causas del comportamiento anómalo de algunos componentes. También podemos incluir la comprobación automática del estado de los servicios en un tablero y probar cuál es la máxima cantidad de nodos Forward que es posible atender con un nodo Master. \par
Otro hecho a destacar es haber podido implementar este proyecto en un entorno complejo y con hardware de alto rendimiento, como es la infraestructura de redes de datos de la Universidad Nacional de Córdoba. Esto nos permitió experimentar y poner a prueba nuestro proyecto bajo condiciones reales y en un ambiente altamente demandante. Fue posible contemplar el desempeño de la solución y poner a punto el sistema. \par
Finalmente pudimos adquirir conocimientos y experiencia, en el área de la seguridad de redes de la información en particular y en ciberseguridad en general. Ensayamos el proceso, vimos sus efectos y consecuencias, de desplegar este tipo de soluciones a gran escala. Fue una gran experiencia haber explorado esta área profesional y haber adquirido aptitudes para desarrollar en el futuro. \par


