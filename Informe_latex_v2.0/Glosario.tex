\chapter*{\Large Glosario}
\addcontentsline{toc}{chapter}{Glosario}

\textbf{CCNA}: acrónimo en inglés de Cisco Certified Network Associate, un certificado de validación profesional emitido por la corporación Cisco para técnicos que operan sus productos. \par

\textbf{CERT}: siglas en inglés de Computer Emergency Response Team, en español equipo de respuesta a incidentes de computadoras. Término registrado comercialmente por la universidad estadounidense de Carnegie Mellon. \par

\textbf{CPU}: siglas en inglés de Central Processing Unit, en español Unidad de Procesamiento Central.\par

\textbf{Creative Commons}: licencia que permite a cualquier usuario copiar, reproducir, adaptar, distribuir, traducir y desarrollar los contenidos multimedia sin costo alguno. La utilización de contenido que se encuentra bajo esta licencia implica reconocer al autor original.\par

\textbf{CSIRT}: siglas en inglés de Computer Security Incident Response Team, en español equipo de respuesta a incidentes de seguridad de computación. Es el equipo de profesionales, sistemas y toda la infraestructura (hardware y software) de detección y respuesta a incidentes de ciberseguridad de una organización.\par

\textbf{DDoS}: siglas en inglés de Denied Distribution Of Service, en español denegación distribuida de servicio. \par

\textbf{DNS}: siglas en inglés de Domain Name Service, en español Servicio de Nombres de Dominio, es un protocolo de red de la capa de aplicación. \par


\textbf{Dirección MAC}: siglas en inglés de Media Access Control, en español control de acceso a medios, es un conjunto de bytes que constituyen la dirección física (única) que identifica a un dispositivo conectado a una red. \par

\textbf{EMCFFAA}: Estado Mayor Conjunto de las Fuerzas Armadas Argentinas. \par

\textbf{Firewall}: Un firewall es un dispositivo basado en un software o hardware o ambos. Bloquea o permite el tráfico de red, basándose en una serie de reglas dinámicas predefinidas y políticas. \par

\textbf{Gbps}: siglas de gigabit por segundo, es una especificación técnica de la medida del ancho de banda y / o velocidad de transmisión, dependiendo del contexto. \par

\textbf{GNU}: acrónimo recursivo en inglés de “GNU is Not Unix”, en español "GNU No es Unix". \par

\textbf{GPL}: siglas en inglés de General Public Licence, en español licencia pública general, es un tipo de licencia GNU. \par

\textbf{GUI}: siglas en inglés de Graphical User Interface, en español Interfaz gráfica de Usuario. \par

\textbf{HIDS}: siglas en inglés de Host Intrusion Detection System, en español sistema de detección de intrusiones en un host o punto final. \par

\textbf{IDS}: siglas en inglés de Intrusion Detection System, en español sistema de detección de intrusiones. Es un componente de software destinado al procesamiento de firmas basadas en la información recolectada del tráfico de red, mediante una sonda colocada en un enlace de la infraestructura de comunicaciones de datos. Los IDS son un componente vital de un CSIRT debido a que realizan la identificación a priori de eventos en el tráfico de datos y su clasificación como un incidente. \par

\textbf{IMAP}:  siglas en inglés de Internet Message Access Protocol. Este protocolo de aplicación, permite a los usuarios acceder a sus e-mails directamente en el servidor y sólo descargar, hacia la máquina local, los mensajes y archivos adjuntos que le resulten de interés. \par

\textbf{Información normalizada}: el objetivo es modificar los mensajes de diferentes fuentes de manera tal que se adapten a un modelo de datos común. \par

\textbf{Infraestructura de IT}: corresponde a la infraestructura de tecnologías de la información (servidores, switches, routers, etc) de una organización. \par

\textbf{IPV4 e IVP6}: siglas en inglés de los protocolos de Internet versiones 4 y 6, respectivamente. \par

\textbf{IPS}: siglas en inglés de Intrusion Protection System, en español sistema de protección de intrusiones. \par

\textbf{Licencia GFDL}: siglas de GNU Free Documentation License. Está orientado a permitir que un manual, un libro de texto o cualquier  otro documento escrito sea libre en el sentido de su difusión, copias, modificaciones y comercialización. \par

\textbf{Licencia AGPL}: siglas de GNU Affero General Public License. Esta licencia asegura los derechos de autor sobre el software y da permisos legales para la copia, distribución y modificaciones del mismo. En caso de modificaciones se debe poner a disposición de la comunidad el código fuente con dichos cambios. \par

\textbf{Licencia APACHE}: licencia de software libre permisiva creada por la Apache Software Foundation. Se diferencia de otros tipos de licencias ya que no exige copyleft en el software donde se aplica. \par

\textbf{Licencia BSD}: siglas en inglés de Berkeley Software Distribution, licencia de software libre desarrollada en dicha universidad homónima de Estados Unidos. \par

\textbf{Linux}: núcleo de código (kernel) abierto de familias de sistemas operativos del mismo nombre, de software libre. \par

\textbf{Log}: equivalente en inglés a “registro” en español. Término utilizado específicamente para registros de datos con un formato definido. \par

\textbf{Malware}: software malicioso diseñado para identificar y / o explotar vulnerabilidades en  los sistemas de  una víctima: sistemas operativos, drivers, cualquier tipo de software, dispositivos, etc. Las consecuencias implican desde el malfuncionamiento del software o dispositivo afectado, robo o pérdida de información, hasta la inutilización total del hardware o sistema infectado. \par

\textbf{MIT}: siglas en inglés de Massachusetts Institute of Technology, universidad de los Estados Unidos cuyo nombre es usado para un tipo de licencia de código libre desarrollada en esa universidad. \par
\textbf{MSSP}: siglas 
en inglés de managed security service provider, en español proveedores de servicios de seguridad gestionados. Empresas que prestan servicios de seguridad informática a organizaciones. \par

\textbf{NIDS}: siglas en inglés de Network Intrusion Detection System, en español sistema de detección de intrusiones a nivel de red. \par

\textbf{NIPS}: siglas en inglés de Network Intrusion Protection System, en español sistema de protección de intrusiones a nivel de red. \par

\textbf{NSM}: siglas en inglés de Network Security Monitoring, en español monitoreo de seguridad de redes. \par

\textbf{RAM}: siglas en inglés de Random Access Memory, en español memoria de acceso aleatorio. \par

\textbf{Ransomware}: Software malicioso, que en un dispositivo puede bloquear la interfaz de usuario o cifrar las información que se encuentra en el disco y posteriormente solicitarle a la víctima un pago para recuperar los datos. \par
\textbf{SaaS}: siglas en inglés de Software as a Service, en español software como servicio, es un modelo de negocio de software a cuyo despliegue y funcionalidades están disponibles a la medida de la demanda del cliente. \par

\textbf{SNMP}: Simple Network Management Protocol (Protocolo simple de administración de red, por sus siglas en ingles). Protocolo que se ubica en el nivel de aplicacion de la pila de red TCP/IP.  \par

\textbf{STDIN}: siglas en ingles de Standard Input, es la entrada estandar de ingreso de datos a un software.\par

\textbf{SYN}: bit usado en el protocolo TCP para indicar la sincronización del número de secuencia al comienzo de una comunicación utilizando el protocolo antes mencionado. \par

\textbf{TCP}: siglas en inglés de Transmission Control Protocol, en español protocolo de control de transmisión. Uno de los protocolos fundamentales en la comunicación de datos. \par

\textbf{UDP}: siglas en inglés de User Datagram Protocol. Es un protocolo que permite la transmisión sin conexión de datagramas en redes basadas en IP. \par

\textbf{VPN}: siglas en inglés de Virtual Private Network, en español red privada virtual. \par
