\documentclass[12pt,twoside]{report}
\usepackage[utf8]{inputenc}
%Para qué los títulos aparezcan en español
\usepackage[spanish]{babel}
\usepackage{csquotes}
\usepackage{array} % para hacer tablas
\usepackage{graphicx} % gráficos
\usepackage{booktabs} %para que las tablas se vean como en los libros y revistas
\usepackage{float} % posiciona las imágenes, tambien usamos placeins
\usepackage{blindtext}
\usepackage{caption} %para los subtítulos de cuadros y figuras
\usepackage{placeins} %para el posicionamiento de imágenes
\usepackage{subfig}
\usepackage[export]{adjustbox}
\usepackage{verbatim} % comentarios
\raggedbottom  %para eliminar los espacios en blanco que deja el espaciado automatico.
%\usepackage{amssymb}
\usepackage[
backend=biber,
sorting=none
]{biblatex} %se puede especificar otra forma de imprimir las citaciones aca
\addbibresource{referencias.bib}

%Margenes según normas APA
\usepackage[a4paper,top=1in, bottom=1in, left=1in, right=1in]{geometry}

% Links a todo el documento y urls
\usepackage{hyperref}
\hypersetup{
    colorlinks=true,
    linkcolor=blue,
    filecolor=magenta,      
    urlcolor=black,
    citecolor=black,
    pdftitle={Sharelatex Example},
    %bookmarks=true,
    pdfpagemode=FullScreen,
}

\begin{document}

    \begin{titlepage}

    \begin{center}

        \begin{tabular}{ p{0.5\textwidth} p{0.5\textwidth} }
        		\centering\includegraphics[height=0.15\textwidth]{./imagenes_caratula/unc_logo.jpg} &
        		\centering\includegraphics[height=0.15\textwidth]{./imagenes_caratula/logo_fcefyn_nuevo.jpg}
        \end{tabular}
        
        \vspace*{30mm}
        \begin{Huge}
            Universidad Nacional de Córdoba \\
        \end{Huge}
        
        \vspace*{10mm}
        \begin{Large}
            Facultad de Ciencias Exactas Físicas y Naturales \\
        \end{Large}
        
        \vspace*{20mm}
        \begin{large}
            Proyecto Integrador: \\
        \end{large}
        
        \vspace*{5mm}
        \begin{LARGE}
            \textbf{Implementación de una solución para gestión de eventos de seguridad de una red de datos} \\
        \end{LARGE}
        
        \vspace*{0.3in}
        \begin{large} \begin{center}
          Autores  
        \end{center}
        \end{large}
        \begin{center}
           \begin{table}[h]
            \resizebox{\textwidth}{!}{%
            \begin{tabular}{|rll|}
            \hline
            \multicolumn{1}{|l}{} & Figueroa Sergio David        & Sepulveda Federico Nicolas       \\
            Correo:               & sergiofigueroa@mi.unc.edu.ar & federico.sepulveda@mi.unc.edu.ar \\
            Matricula:            & 36355236                     & 35037929                         \\
            Telefono:             & 3512941114               & 2994165771                   \\ \hline
            \end{tabular}%
            }
            \end{table}
        \end{center}
        \vspace*{0.3mm}
        \rule{100mm}{0.1mm}\\
        \vspace*{0.3mm}
        \begin{large}
            Director: \\
            Mgter. Ing. Miguel Ángel Solinas \\
        \end{large}
        

    \end{center}

\end{titlepage}

    \chapter*{\Large Resumen}
\addcontentsline{toc}{chapter}{Resumen}

Los servicios digitales de procesamiento y almacenamiento de información se transformaron en la columna vertebral de organizaciones de todo tipo, ya que estas volcaron masivamente su actividad a Internet en la última década. Esto genera un sinnúmero de oportunidades para eventuales atacantes, que buscan detectar y explotar las vulnerabilidades de los sistemas en los cuales se apoya la infraestructura de las organizaciones. \par
El proceso de monitoreo de la seguridad de una red de datos compleja requiere recopilar diferentes tipos de datos para detectar, verificar y contener acciones ofensivas. Es necesario contar con un sistema SIEM que centralice la información proveniente de múltiples fuentes, distribuidas a lo largo de toda la red de datos. En este trabajo se propuso, implementó y desplegó una solución SIEM en la Universidad Nacional de Córdoba.\par

















    % Índice general
    \tableofcontents
    % Índice de figuras
    \listoffigures
    % Índice de tablas
    \listoftables
    % Agradecimientos
    \chapter*{\Large Agradecimientos}
\addcontentsline{toc}{chapter}{Agradecimientos}

    % Glosario
    \chapter*{Glosario}
\addcontentsline{toc}{chapter}{Glosario}

\textbf{CCNA}: acrónimo en inglés de Cisco Certified Network Associate, un certificado de validación profesional emitido por la corporación Cisco para técnicos que operan sus productos. \par

\textbf{CERT}: siglas en inglés de Computer Emergency Response Team, en español equipo de respuesta a incidentes de computadoras. Término registrado comercialmente por la universidad estadounidense de Carnegie Mellon. \par

\textbf{CPU}: siglas en inglés de Central Processing Unit, en español Unidad de Procesamiento Central.\par

\textbf{Creative Commons}: licencia que permite a cualquier usuario copiar, reproducir, adaptar, distribuir, traducir y desarrollar los contenidos multimedia sin costo alguno. La utilización de contenido que se encuentra bajo esta licencia implica reconocer al autor original.\par

\textbf{CSIRT}: siglas en inglés de Computer Security Incident Response Team, en español equipo de respuesta a incidentes de seguridad de computación. Es el equipo de profesionales, sistemas y toda la infraestructura (hardware y software) de detección y respuesta a incidentes de ciberseguridad de una organización.\par

\textbf{DDoS}: siglas en inglés de Denied Distribution Of Service, en español denegación distribuida de servicio. \par

\textbf{DNS}: siglas en inglés de Domain Name Service, en español Servicio de Nombres de Dominio, es un protocolo de red de la capa de aplicación. \par


\textbf{Dirección MAC}: siglas en inglés de Media Access Control, en español control de acceso a medios, es un conjunto de bytes que constituyen la dirección física (única) que identifica a un dispositivo conectado a una red. \par

\textbf{EMCFFAA}: Estado Mayor Conjunto de las Fuerzas Armadas Argentinas. \par

\textbf{Firewall}: Un firewall es un dispositivo basado en un software o hardware o ambos. Bloquea o permite el tráfico de red, basándose en una serie de reglas dinámicas predefinidas y políticas. \par

\textbf{Gbps}: siglas de gigabit por segundo, es una especificación técnica de la medida del ancho de banda y / o velocidad de transmisión, dependiendo del contexto. \par

\textbf{GNU}: acrónimo recursivo en inglés de “GNU is Not Unix”, en español "GNU No es Unix". \par

\textbf{GPL}: siglas en inglés de General Public Licence, en español licencia pública general, es un tipo de licencia GNU. \par

\textbf{GUI}: siglas en inglés de Graphical User Interface, en español Interfaz gráfica de Usuario. \par

\textbf{HIDS}: siglas en inglés de Host Intrusion Detection System, en español sistema de detección de intrusiones en un host o punto final. \par

\textbf{IDS}: siglas en inglés de Intrusion Detection System, en español sistema de detección de intrusiones. Es un componente de software destinado al procesamiento de firmas basadas en la información recolectada del tráfico de red, mediante una sonda colocada en un enlace de la infraestructura de comunicaciones de datos. Los IDS son un componente vital de un CSIRT debido a que realizan la identificación a priori de eventos en el tráfico de datos y su clasificación como un incidente. \par

\textbf{IMAP}:  siglas en inglés de Internet Message Access Protocol. Este protocolo de aplicación, permite a los usuarios acceder a sus e-mails directamente en el servidor y sólo descargar, hacia la máquina local, los mensajes y archivos adjuntos que le resulten de interés. \par

\textbf{Información normalizada}: el objetivo es modificar los mensajes de diferentes fuentes de manera tal que se adapten a un modelo de datos común. \par

\textbf{Infraestructura de IT}: corresponde a la infraestructura de tecnologías de la información (servidores, switches, routers, etc) de una organización. \par

\textbf{IPV4 e IVP6}: siglas en inglés de los protocolos de Internet versiones 4 y 6, respectivamente. \par

\textbf{IPS}: siglas en inglés de Intrusion Protection System, en español sistema de protección de intrusiones. \par

\textbf{Licencia GFDL}: siglas de GNU Free Documentation License. Está orientado a permitir que un manual, un libro de texto o cualquier  otro documento escrito sea libre en el sentido de su difusión, copias, modificaciones y comercialización. \par

\textbf{Licencia AGPL}: siglas de GNU Affero General Public License. Esta licencia asegura los derechos de autor sobre el software y da permisos legales para la copia, distribución y modificaciones del mismo. En caso de modificaciones se debe poner a disposición de la comunidad el código fuente con dichos cambios. \par

\textbf{Licencia APACHE}: licencia de software libre permisiva creada por la Apache Software Foundation. Se diferencia de otros tipos de licencias ya que no exige copyleft en el software donde se aplica. \par

\textbf{Licencia BSD}: siglas en inglés de Berkeley Software Distribution, licencia de software libre desarrollada en dicha universidad homónima de Estados Unidos. \par

\textbf{Linux}: núcleo de código (kernel) abierto de familias de sistemas operativos del mismo nombre, de software libre. \par

\textbf{Log}: equivalente en inglés a “registro” en español. Término utilizado específicamente para registros de datos con un formato definido. \par

\textbf{Malware}: software malicioso diseñado para identificar y / o explotar vulnerabilidades en  los sistemas de  una víctima: sistemas operativos, drivers, cualquier tipo de software, dispositivos, etc. Las consecuencias implican desde el malfuncionamiento del software o dispositivo afectado, robo o pérdida de información, hasta la inutilización total del hardware o sistema infectado. \par

\textbf{MIT}: siglas en inglés de Massachusetts Institute of Technology, universidad de los Estados Unidos cuyo nombre es usado para un tipo de licencia de código libre desarrollada en esa universidad. \par
\textbf{MSSP}: siglas 
en inglés de managed security service provider, en español proveedores de servicios de seguridad gestionados. Empresas que prestan servicios de seguridad informática a organizaciones. \par

\textbf{NIDS}: siglas en inglés de Network Intrusion Detection System, en español sistema de detección de intrusiones a nivel de red. \par

\textbf{NIPS}: siglas en inglés de Network Intrusion Protection System, en español sistema de protección de intrusiones a nivel de red. \par

\textbf{NSM}: siglas en inglés de Network Security Monitoring, en español monitoreo de seguridad de redes. \par

\textbf{RAM}: siglas en inglés de Random Access Memory, en español memoria de acceso aleatorio. \par

\textbf{Ransomware}: Software malicioso, que en un dispositivo puede bloquear la interfaz de usuario o cifrar las información que se encuentra en el disco y posteriormente solicitarle a la víctima un pago para recuperar los datos. \par
\textbf{SaaS}: siglas en inglés de Software as a Service, en español software como servicio, es un modelo de negocio de software a cuyo despliegue y funcionalidades están disponibles a la medida de la demanda del cliente. \par

\textbf{SNMP}: Simple Network Management Protocol (Protocolo simple de administración de red, por sus siglas en ingles). Protocolo que se ubica en el nivel de aplicacion de la pila de red TCP/IP.  \par

\textbf{STDIN}: siglas en ingles de Standard Input, es la entrada estandar de ingreso de datos a un software.\par

\textbf{SYN}: bit usado en el protocolo TCP para indicar la sincronización del número de secuencia al comienzo de una comunicación utilizando el protocolo antes mencionado. \par

\textbf{TCP}: siglas en inglés de Transmission Control Protocol, en español protocolo de control de transmisión. Uno de los protocolos fundamentales en la comunicación de datos. \par

\textbf{UDP}: siglas en inglés de User Datagram Protocol. Es un protocolo que permite la transmisión sin conexión de datagramas en redes basadas en IP. \par

\textbf{VPN}: siglas en inglés de Virtual Private Network, en español red privada virtual. \par

    % A partir de acá comienza la numeración de los capítulos
    \chapter{\Large Introducción}
    \begin{section}{Sistema SIEM}
        Este proyecto consiste en la implementación de un Sistema de Administración de Eventos y Seguridad de la Información (SIEM, por sus siglas en inglés) para la Universidad Nacional de Córdoba. Un SIEM consiste en varias herramientas como bases de datos, filtros para normalizar la información, tablero para visualizar los datos y generador de alertas entre otras. Por otro lado, tenemos el monitoreo de la red en tiempo real que utiliza un Sistema de Detección de Intrusiones (IDS, por sus siglas en inglés). Este último, envía los datos ya procesados al SIEM para que los almacene en la base de datos. \par
        Además, como se pretende que el SIEM funcione dentro de un Equipo de Respuesta a Incidentes de Seguridad Informática (CSIRT, por sus siglas en inglés) se necesita integrar un gestor de incidentes. Este último, sirve para tener un registro de los incidentes ocurridos, permite administrar las tareas del equipo de analistas, compartir y solicitar información con otros CSIRT entre otras funciones.
    \end{section}
    \begin{section}{Objetivo general}
        El objetivo de este proyecto integrador es el desarrollo e implantación de un sistema SIEM dentro del proyecto general de la creación del CSIRT de la Universidad Nacional de Córdoba, con el fin de otorgar al mencionado centro de respuesta, el instrumento capaz de obtener, analizar y presentar datos sobre las amenazas detectadas por los demás subsistemas del CSIRT.
    \end{section}
    
    \begin{section}{Motivación}
        La tecnología y la digitalización de la información convierten a los datos en un activo muy importante de las organizaciones y de los individuos en general. Es fundamental saber cómo proteger los datos para evitar ser víctima de un ciberdelito o parte involuntaria de una ciber operación a gran escala. A pesar que actualmente las técnicas de seguridad hacia los datos y la infraestructura de redes están en auge, las herramientas de seguridad como \textit{firewalls}, IDS y otras que permiten prevenir ataques informáticos no son suficientes para mitigar y tener un seguimiento de actividades maliciosas o potencialmente maliciosas para lograr fortalecer la infraestructura y prevenir futuros incidentes. Resulta necesario contar con un sistema global que permita integrar un variado conjunto de utilidades que brindan soluciones puntuales y específicas, para crear una defensa inteligente y eficiente de los activos de información de una organización. \par
        Actualmente la infraestructura de red y los sistemas asociados conviven en un ambiente de saturación de la información, que implica un alto costo de procesamiento. Esto ultimo, constituye un desafío constante para los sistemas encargados de la optimización de los recursos de hardware y software con los que cuenta la infraestructura, tales como uso de CPU y memorias RAM de \textit{routers}, \textit{switches} y servidores. Por otro lado, la retención de un ingente volumen de datos generados por el exponencial y siempre creciente tráfico de la red, amenaza constantemente con el colapso de los medios disponibles sin importar su capacidad de almacenamiento. Estos problemas, entre otros, configuran una avalancha constante de información que sería imposible de analizar (siquiera una parte de ella) en un momento determinado utilizando métodos que impliquen el procesamiento en bruto.  \par
        En esta situación, sería imposible distinguir un evento puntual y nocivo dentro de esta cantidad gigantesca de información que se genera permanentemente en la red, de un evento normal o de tráfico legítimo y en caso de identificar un potencial incidente, este tendría unas probabilidades muy altas de ser un falso positivo. Esto último es característico de los sistemas basados en el análisis de firmas, como los IDS, IPS o antivirus. Es necesario diseñar, desarrollar, implementar, configurar y probar un sistema capaz de orquestar un gran abanico de herramientas diseñadas cada una con un objetivo puntual, combinando las capacidades de todos sus subsistemas para identificar eficientemente las amenazas reales y responder en consecuencia, minimizando los falsos positivos y daños colaterales.

    \end{section}
    \chapter{Marco Teórico}
    \begin{section}{Presentación}
        Las infracciones a las políticas de seguridad y los ataques han concentrado la atención sobre las capacidades de detección, investigación y mitigación de incidentes de seguridad de la información en  las organizaciones. Si bien no siempre es posible evitar un incidente de seguridad, es necesario detectar y responder rápidamente para minimizar el daño. Para ello, es preciso realizar inversiones inteligentes basadas en un plan de seguridad que comprenda la realidad y necesidades específicas de la organización, ya que un gran monto de dinero o equipos adquiridos por si mismos no garantizan una mayor protección. \par
        Este plan debe incluir personal especializado, procedimientos e infraestructura  adaptados a la organización, con una gestión de objetivos a cumplir a corto, mediano y largo plazo. \par
        Para las organizaciones que no cuentan con una capacidad de manejo de incidentes, la creación desde cero de un Computer Security Incident Response Team (CSIRT) puede ser un proceso complejo y costoso. Sin embargo, no es necesario una gran inversión para obtener las capacidades elementales ofrecidas por un CSIRT, ya que es posible desarrollar una solución específica y a escala de la organización. \par
        Una vez identificadas las necesidades de la organización, el proceso de creación del CSIRT requiere de la creación, colaboración y comunicación entre los tres pilares que lo componen: el personal, la tecnología y los procesos, como se muestra en la Figura \ref{fig:pilares}. \par
        \begin{figure}[H]
            \centering
            \includegraphics[width=1\textwidth]{./marco_teorico_imagenes/figura_1_pilares.png}
            \caption{Pilares de un CSIRT}
            \label{fig:pilares}
        \end{figure}
        %\par
        El CSIRT debe tener una perspectiva flexible y escalable para mantener el ritmo de las tácticas de los adversarios, acompañando el crecimiento y evolución de la organización. \par
    \end{section}
    
   \begin{section}{Personal}  
   En cuanto al personal, estos comprenden tanto a los encargados de dar respuesta a los incidentes como a los analistas del CSIRT. Si bien la propia organización puede designar a sus integrantes para asumir estas funciones, existen otras alternativas como la tercerización mediante empresas especializadas que proveen el servicio de Managed Security Service Provider (MSSP) o contratar especialistas en respuesta a incidentes en el caso de una emergencia o un problema complejo. Otra vía consiste en la creación de equipos híbridos compuestos por personal perteneciente a la organización y especialistas externos. \par
    De acuerdo a una encuesta del SANS Institute del año 2014 \cite{sans_1}, el 61\% de las organizaciones relevadas manifestaron haber recurrido a personal de emergencia para cubrir incidentes críticos y el 58 \% tenía un equipo de respuesta propio. Por lo que las organizaciones no siempre cubren sus necesidades con miembros de su propio personal y en algunos casos las tareas recaen por completo en los servicios de terceros. Esto se debe a que, sin importar la estructura del equipo, el personal de un CSIRT debe contar con el entrenamiento necesario para tratar con los cambios en las amenazas a las que se enfrenta. En el Cuadro \ref{table:1} se muestran las responsabilidades y la formación requerida para cada uno de los integrantes de un CSIRT. \par
    
    \begin{table}%[ht]
    \centering
        \begin{tabular}{ | m{10em} | m{16em}| m{11em} | } 
            \hline
            Título profesional & Tarea & Entrenamiento requerido \\ 
            \hline
            Nivel 1 - Analista de alertas & Supervisa continuamente la cola de alertas, monitorea el estado de los sensores y los puntos finales, clasifica las alertas de seguridad y recopila los datos necesarios para iniciar el trabajo de Nivel 2. & Procedimientos de triage de alerta y detección de intrusos. Gestión de redes, información de seguridad y eventos. Capacitación en investigación basada en host. \\ 
            \hline
            Nivel 2 - Analista de respuesta a incidentes & Realiza un análisis profundo de incidentes al correlacionar datos de varias fuentes y determina si un sistema crítico o un conjunto de datos se ha visto afectado. Asesora sobre su remediación. & Análisis avanzado de forensia de redes y basado en host. Procedimientos de respuesta a incidentes, revisiones de registros, evaluación básica de malware e inteligencia de amenazas. \\ 
            \hline
            Nivel 3 - Especialista en la materia & Se trata de un conjunto de especialistas que cubren distintas áreas de un CSIRT. 
            Actúan como “cazadores” de amenazas, sin esperar que se intensifiquen los incidentes. Se encuentra estrechamente involucrado en el desarrollo, ajuste e implementación de análisis de detección de amenazas.
             & Entrenamiento avanzado en detección de anomalías. Entrenamiento específico en herramientas para la agregación y análisis de datos e inteligencia de amenazas. 
            Poseen un conocimiento profundo en áreas como redes, puntos finales, inteligencia de amenazas, forensia e ingeniería inversa de malware, así como la infraestructura de IT subyacente.
            \\ 
             \hline
            Director del CSIRT & Administra recursos para incluir personal, presupuesto, programación de turnos y estrategias para cumplir con los acuerdos de nivel de servicio. Se comunica con la gerencia y sirve como persona de contacto en el caso de incidentes críticos. Proporciona una dirección general para el CSIRT. & Gestión de proyectos, formación en gestión de respuesta a incidentes, habilidades generales de gestión de personas y comunicación institucional.  \\
            \hline %linea final de tabla
        \end{tabular}
        \caption{Integrantes de un CSIRT y sus funciones}
        \label{table:1}
    \end{table}
    
   \end{section}
   
   \begin{section}{Procesos}  
   
   \end{section}
   \begin{section}{Tecnología}  
        \begin{subsection}{Agregando contexto a los incidentes}
        La incorporación de inteligencia de amenazas y otras informaciones de contexto tales como activos e identidades, contribuye al proceso de investigación del analista de un CSIRT. En determinados casos, la información inicial que está asociada a una alerta puede ser muy limitada, por ejemplo la dirección IP del punto final sospechoso es insuficiente por sí sola para tomar una decisión. \par
        Para que los analistas puedan investigar un incidente, generalmente necesitan más información, por ejemplo los nombres del dueño y de dominio de la máquina, registros DHCP para mapear la IP con el host al momento del incidente, etc. Si el sistema de monitoreo incorpora información de identidad y de los activos de información, entre otros datos de contexto, le permitirá al analista ahorrar tiempo y esfuerzo para priorizar los incidentes y elaborar la respuesta más apropiada.\par

        \end{subsection}
        \begin{subsection}{Agregando contexto a los incidentes}
        \end{subsection}
        
        \begin{subsection}{Definición de conductas normales}
        \end{subsection}
        
        \begin{subsection}{Inteligencia de amenazas}
        \end{subsection}
        
        \begin{subsection}{Obstáculos para el manejo eficiente de incidentes del CSIRT}
        \end{subsection}
   \end{section}
      
   \begin{section}{Ámbitos de actuación de los CSIRT}
        \begin{subsection}{Estado de la ciberseguridad en Argentina} 
            \begin{subsubsection}{Demanda de ciberseguridad Argentina}
            \end{subsubsection}
        \end{subsection}
   \end{section}

    \begin{section}{SIEM: Definición y funciones}
    
    \end{section}
    \begin{section}{Soluciones disponibles}
    
    
        \begin{subsection}{Soluciones comerciales}
        
        \end{subsection}
        \begin{subsection}{Soluciones gratuitas y de código abierto}
        
        
            \begin{subsubsection}{AlienVault OSSIM}
            
            \end{subsubsection}
            \begin{subsubsection}{Graylog}
            
            \end{subsubsection}
            \begin{subsubsection}{Elastic Stack}
            
            \end{subsubsection}
            \begin{subsubsection}{Security Onion}
            
            \end{subsubsection}
            
        \end{subsection} 
    \end{section}
    \begin{section}{Corolario}
    
    \end{section}
            
            
            
            

    \chapter{Descripción de Requerimientos}
    Con el objetivo de desplegar un sistema SIEM capaz de soportar los requerimientos funcionales y no funcionales, es necesario definir el entorno en el que operará la plataforma. \par
    Para esto se requiere, en primer lugar describir la topología de la red de la organización: realizar un relevamiento de las conexiones  de la infraestructura de red interna de la unidad central, las de sus dependencias y la red entre las unidades geográficamente distribuidas si las hubiera. Debe incluirse la topología de las conexiones de salida a Internet. \par
    %En segundo lugar será necesario inventariar los activos de la organización: se requerirá una investigación y un relevamiento de los activos con los que cuentan las infraestructuras de red y de datos a fin de clasificarlos. \par
    Las tareas de relevamiento anteriormente descritas proporcionarán un entendimiento acabado y profundo de la situación en la que se encuentra la infraestructura. Como resultado, será posible identificar puntos críticos a tener en consideración y como consecuencia, elegir la solución que mejor se ajuste a las necesidades de la organización. \par

    \begin{section}{Requerimientos funcionales del SIEM}
    \begin{enumerate}
        \item Recolectar y almacenar datos de incidentes de seguridad en la infraestructura de la red corporativa.
        \item Recolectar y almacenar información contextual y asociada a los activos vinculados  al incidente.
        \item Visualizar las alertas en un tablero de mando. 
        \item Implementar un sistema de envío de alertas de seguridad que notifique a los responsables de activos de información afectados.
        \item Definir un criterio para priorizar alertas.
        \item Implementar un sistema de correlación de alertas de seguridad.
    \end{enumerate}
        
    \end{section}
    
    \begin{section}{Requerimientos no funcionales}
    \begin{enumerate}
        \item La solución propuesta debe utilizar software libre.
        \item El sistema operativo base debe ser tipo Unix y abierto.
        \item La arquitectura de la solución debe ser escalable a demanda de la organización.
        \item Se requiere un despliegue automatizado de la solución.
    \end{enumerate}

    \end{section}
    
    \begin{section}{Análisis de riesgo}
    En primer lugar, será seleccionada una plataforma cuyo código sea libre y abierto. Posteriormente será elegido un sistema operativo libre, tipo Unix, que sea compatible con la solución escogida. Se tendrá en cuenta el desarrollo de una arquitectura de despliegue que contemple la escalabilidad horizontal de la solución para adaptarse a las necesidades de la organización. \par
    Se adaptará la solución para recolectar y almacenar los datos pertinentes a los incidentes de seguridad que ocurren en la red corporativa, así como la información de contexto de los activos de información que se ven afectados.  Luego de haber recibido y almacenado los datos, se configurará la solución para visualizar la información disponible. Esto permitirá implementar un sistema de envío de alertas que notifiquen a los responsables de los activos de información que se viesen comprometidos. \par
    Finalmente, se dispondrá de un sistema que permita correlacionar y filtrar alertas, en base a políticas a definir que incluyan las categorías de eventos y sus prioridades asociadas.\par
	%Finalmente se procederá a automatizar el proceso completo que comprende la implementación, despliegue y configuración de la solución junto a sus componentes asociados, mediante el uso de herramientas de automatización de tecnologías de la información. \par
    \end{section}
    
    
    \chapter{\Large Descripción de Security Onion}
    \begin{section}{Security Onion como sistema de gestión de eventos}
    \begin{figure}[H]
        \centering
        \includegraphics[width=0.7\textwidth]{./descripcion_sonion_imagenes/figura_15_logo_sonion.png}
        \caption{logo de Security Onion\cite{sonion}}
        \label{fig:logo_sonion}
    \end{figure}
        La elección de Security Onion (versión 16.04) como plataforma se justificó en su naturaleza de código abierto y por sus características destacables respecto de otras soluciones libres, como el soporte de una activa comunidad, el desarrollo continuo de mejoras, actualizaciones y correcciones, su capacidad polimórfica y funcional de actuar como IDS, plataforma SIEM o cluster de almacenamiento. Esto permitió desarrollar distintas arquitecturas de una manera fácil y asistida para el despliegue y la consiguiente optimización de los recursos de hardware y de red. \par
        Otras de las propiedades destacables es la capacidad de integración directa con un conjunto casi universal de los sistemas IDS disponibles, tanto libres como propietarios. Security Onion también incluye un paquete de configuraciones iniciales predefinidas para la infraestructura inicial del sistema, tales como el almacenamiento, normalización y gestión de logs, sistemas IDS y de gestión de usuarios, entre otros. \par
    \end{section}
    \begin{section}{Arquitectura del sistema de gestión de eventos}
    \end{section}
    \begin{subsection}{Arquitectura de alto nivel}
    \label{tipo_nodos}
    \begin{figure}[H]
        \centering
        \includegraphics[width=0.7\textwidth]{./descripcion_sonion_imagenes/figura_16_arq_alto_nivel_sonion.png}
        \caption{Security Onion\cite{sonion}: Arquitectura de alto nivel}
        \label{fig:arq_top_sonion}
    \end{figure}
    En la Figura \ref{fig:arq_top_sonion} se observa la distribución de Security Onion y el flujo de datos entre sus componentes principales (la pila Elastic) y secundarios (Curator \cite{curator}, ElastAlert\cite{elasalert}, freqServer\cite{freqserver} y domainStats\cite{domainstat}). Se puede apreciar la conexión con los sistemas de detección IDS como Bro\cite{zeek}, Snort\cite{snort}, Suricata\cite{suricata}, Syslog, etc. Se distinguen también los enlaces con los puntos de administración de los analistas del CSIRT y con los servicios web externos para el envío y recepción de alertas, notificaciones, análisis de tráfico, entre otros. Un punto a destacar es que la pila Elastic se encuentra desplegada en contenedores Docker\cite{docker}. 
    \end{subsection}
    \pagebreak
    \begin{subsubsection}{Tipo de Nodos}
        \begin{itemize}
          \item Nodo Master: este nodo ejecuta su propia copia de la base de datos Elasticsearch, con la que gestiona las búsquedas a través del cluster y estructura a otros nodos en el momento de su despliegue. Lo anterior implica que puede realizar las configuraciones necesarias para los nodos de los tipos “densos” y los de almacenamiento, pero no los de sensores o Forward, por carecer estos últimos de una pila Elastic. Este nodo permite a un analista conectarse mediante un enlace de administración para realizar consultas de los datos.
          \begin{itemize}
              \item  Este nodo contiene los siguientes componentes:
              \begin{itemize}
                  \item Elasticsearch \cite{elastic}
                  \item Logstash \cite{elastic}
                  \item Kibana \cite{elastic}
                  \item Curator \cite{curator}
                  \item ElastAlert \cite{elasalert}
                  \item Redis \cite{redis}
                  \item Wazuh \cite{wazuh} / OSSEC \cite{ossec}
                  \item Sguild \cite{sguil}
              \end{itemize}
          \end{itemize}
          Elasticsearch \cite{elastic}, Kibana \cite{elastic} y Logstash \cite{elastic} son componentes de la pila Elastic, que se tratarán en la siguiente sección junto a ElastAlert \cite{elasalert}. El objetivo de Curator \cite{curator} y Redis \cite{redis} es administrar y optimizar las bases de datos de los nodos de almacenamiento; Wazuh \cite{wazuh} es un IDS y Security Onion lo utiliza para el monitoreo de sí mismo, configurando un sistema HIDS ad hoc, aunque es posible desplegarlo en otros nodos o puntos de interés. Sguild \cite{sguil} permite consultar eventos de una base de datos MySQL desde dentro de Security Onion y muestra los resultados en una GUI. Además, actúa como intermediario de otros componentes secundarios como Squert \cite{squert}, del que detallaremos sus funciones y comportamiento en una sección posterior. 
          \item Nodos Forward: este nodo cumple la función de procesar el tráfico y reenviar los resultados al nodo master. Los logs generados por Snort / Suricata y Bro son enviados mediante syslog a Logstash en el nodo Master, utilizando un túnel ssh, donde finalmente son guardados en la base de datos Elasticsearch, donde pueden ser reenviados a los nodos de almacenamiento. Los logs pueden ser consultados a través de una búsqueda en el cluster.
            \begin{itemize}
               \item Los componentes de un nodo Forward son:
               \begin{itemize}
                   \item Zeek \cite{zeek} (sucesor de Bro)
                   \item Snort \cite{snort} / Suricata \cite{suricata}
                   \item Netsniff-ng \cite{netsniff-ng}
                   \item Wazuh \cite{wazuh} / OSSEC \cite{ossec}
                   \item Syslog-ng \cite{syslog-ng}
               \end{itemize}
            \end{itemize}
            Zeek, Snort / Suricata y Netsniff-ng son procesadores de tráfico (IDS), donde Snort y Suricata serán tratados en una sección posterior. Syslog-ng es utilizado para recolectar logs de los IDS y enviarlos al Logstash del master, donde serán procesados y tratados antes de ser escritos en Elasticsearch.
            \item Nodos Pesados (Heavy Nodes): Es un nodo híbrido entre el nodo Forward y el nodo Master, que incluye todos los componentes del nodo Forward, además de una instancia completa de la pila Elastic. Los nodos pesados envían los resultados de las consultas de su instancia local de Elasticsearch a las solicitudes realizadas por el nodo master mediante un túnel de autossh.
            \begin{itemize}
                \item Los componentes de este nodo son:
                \begin{itemize}
                    \item Elasticsearch
                    \item Logstash
                    \item Curator
                    \item Zeek
                    \item Snort / Suricata
                    \item Netsniff-ng
                    \item Wazuh / OSSEC
                    \item Syslog-ng (envía los logs a la instancia local de Logstash)
                \end{itemize}
            \end{itemize}
            \item Nodos de almacenamiento (Storage Node): su objetivo es extender las capacidades de almacenamiento y procesamiento del nodo Master. Estos nodos despliegan una instancia local de la pila Elastic. De manera análoga a los nodos pesados, cuando se realiza una consulta por parte de la instancia Elasticsearch del nodo Master, esta es procesada por la instancia local del nodo de almacenamiento y devuelta por un túnel autossh.
            \begin{itemize}
                \item Los componentes del nodo de almacenamiento son:
                \begin{itemize}
                    \item Elasticsearch
                    \item Logstash
                    \item Curator
                    \item Wazuh / OSSEC
                \end{itemize}
            \end{itemize}
        \end{itemize}
    \end{subsubsection}
    \pagebreak
    \begin{subsubsection}{Tipos de Arquitectura}
    \label{tipos_de_arquitectura}
      La versatilidad de disponer de múltiples arquitecturas permite adaptar la plataforma a las necesidades de la organización en la que se implante. A continuación, se describen cada una de las opciones posibles:
    \begin{itemize}
         \item Arquitectura monolítica: Consiste en un único servidor que ejecuta simultáneamente los componentes centrales o propios de un nodo Master y los de un nodo sensor. Es un modo híbrido y concentrado que no se recomienda para enlaces de red de alto rendimiento por los elevados requerimientos de hardware necesarios. 
         Este tipo de arquitectura se recomienda para propósitos de pruebas en laboratorio y en entornos de baja demanda de tráfico de red.
        \begin{figure}[H]
            \centering
            \includegraphics[width=0.7\textwidth]{./descripcion_sonion_imagenes/figura_17_arq_monolitica_sonion.png}
            \caption{Arquitectura monolítica de Security Onion\cite{sonion}}
            \label{fig:arq_monolitica_sonion}
        \end{figure}
        \FloatBarrier
        \pagebreak
        \item Arquitectura densamente distribuida: consiste en uno o más nodos pesados (Heavy Nodes) conectados a un nodo Master. Solo se recomienda en el caso de que no sea posible desplegar una arquitectura distribuida, ya que tiene las mismas deficiencias de rendimiento de la arquitectura monolítica y no es apropiado para entornos de producción y/o enlaces de red de alta velocidad.
        \begin{figure}[H]
            \centering
            \includegraphics[width=0.7\textwidth]{./descripcion_sonion_imagenes/figura_18_arq_densa_sonion.png}
            \caption{Arquitectura densamente distribuida de Security Onion\cite{sonion}}
            \label{fig:arq_densa_sonion}
        \end{figure}
        \FloatBarrier
        \pagebreak
        \item Arquitectura Distribuida: consiste en un servidor Master, uno o más nodos Forward y uno o más nodos de almacenamiento (Storage Nodes). Es el tipo de despliegue recomendado en términos de eficiencia de requerimientos de hardware, balance de la carga y almacenamiento de datos y optimización general de los recursos disponibles en la organización. 
        \begin{figure}[H]
            \centering
            \includegraphics[width=0.7\textwidth]{./descripcion_sonion_imagenes/figura_19_arq_distribuida_sonion.png}
            \caption{Arquitectura distribuida de Security Onion\cite{sonion}}
            \label{fig:arq_distribuida_sonion}
        \end{figure}
     \end{itemize}
   \end{subsubsection}
   \pagebreak
  
   \begin{section}{Recibiendo, procesando y visualizando eventos: La pila Elastic}
   En la Figura \ref{fig:figura_20_conexion_comp_elastic} se observa, que luego de recolectar los datos provenientes de múltiples fuentes, es necesario normalizarlos y agregarlos a la base de datos; estas tareas son llevadas a cabo por los componentes de la pila Elastic (versión 6.8.0), en este caso Logstash y Elasticsearch, respectivamente. 

   \begin{figure}[H]
        \centering
        \includegraphics[width=0.45\textwidth]{./descripcion_sonion_imagenes/figura_20_conexion_comp_elastic.png}
        \caption{Conexión de componentes Elastic}
        \label{fig:figura_20_conexion_comp_elastic}
    \end{figure}
        
   \end{section}
   %\pagebreak
   \begin{subsection}{Logstash}
        Logstash es una tubería (en adelante “pipeline”) de procesamiento de datos del lado del servidor que ingesta datos de una multitud de fuentes, los transforma y luego los envía a su destino. Las fuentes de entrada admitidas por logstash son extremadamente amplias, como por ejemplo: syslog, STDIN, TCP, UDP, SNMP, IMAP, entre otras. Posteriormente, Logstash toma los datos sin estructura y los normaliza para crear conjunto ordenado mediante la identificación y conversión de la información a un formato común. Para realizar la tarea anterior, dispone de una gran variedad de filtros que facilitan el procesamiento general, independientemente de la fuente de datos. En este proyecto se utilizó el plugin grok (versión 4.0.4) \cite{grok} como filtro de las fuentes de información. Con los datos ya normalizados, es posible darles un formato específico para un destino en particular, ya que Logstash admite múltiples destinos para la etapa final del pipeline; desde una base de datos, archivos finales o servicios web. Security Onion, por defecto, almacena estos datos normalizados en un formato JSON en la misma pila Elastic, es decir la base de datos Elasticsearch.
   \end{subsection}
   
   \begin{subsection}{Elasticsearch}
        Elasticsearch es una base de datos del tipo NoSQL distribuida y orientada al almacenamiento de documentos. Los datos normalizados provenientes de Logstash son documentos almacenados en índices en Elasticsearch. Cada índice está compuesto por uno o más shards (fragmento), por lo tanto un shard es un subconjunto de documentos, siendo el elemento básico de Elasticsearch y el que permite la escalabilidad del mismo. Un shard es también una instancia de un “índice de Lucene \cite{lucene}", que indexa y almacena un documento en un segmento. Lucene es una librería desarrollada en Java (versión 8) para hacer búsquedas en una base de datos, constituyéndose en un motor de búsqueda que indexa y administra consultas en un conjunto de segmentos. La Figura \ref{fig:figura_21_arq_alm_elasticsearch} muestra la arquitectura de alto nivel del almacenamiento en Elasticsearch.
        
        \begin{figure}[H]
            \centering
            \includegraphics[width=0.7\textwidth]{./descripcion_sonion_imagenes/figura_21_arq_alm_elasticsearch.png}
            \caption{Arquitectura de almacenamiento en Elasticsearch}
            \label{fig:figura_21_arq_alm_elasticsearch}
        \end{figure}
    \end{subsection}
    %\pagebreak
    \begin{subsection}{Kibana}
     Todos los datos almacenados en Elasticsearch pueden ser visualizados por Kibana, una interfaz gráfica perteneciente a la pila Elastic. Kibana permite visualizar los datos en gráficos circulares, de barras, histogramas, etc e interactuar con ellos; también es posible realizar análisis de ubicación cuando se disponen de los metadatos correspondientes mediante el complemento Elastic Maps, realizar análisis de series temporales de una manera rápida y sencilla, dispone de herramientas de inteligencia artificial, que mediante aprendizaje no supervisado permite detectar anomalías y patrones mediante las proyecciones sobre los datos. \par
     Otra de sus características es  la posibilidad de realizar gráficos de correlación y entrecruzamiento, seleccionando campos de interés y filtros lógicos creados por el usuario. Es de destacar que para algunas de estas características es necesario la instalación de plugins complementarios y aunque en su inmensa mayoría son gratuitos, algunos pueden ser pagos ya que utilizan servicios web de la nube de los desarrolladores. 
     \begin{figure}[H]
        \centering
        \includegraphics[width=1\textwidth]{./descripcion_sonion_imagenes/figura_22_capt_kibana.png}
        \caption{Captura de pantalla de Kibana\cite{elastic}}
        \label{fig:captura_kibana}
        \end{figure}
   \end{subsection}
   \begin{section}{Elastic, ElastAlert, TheHive y Cortex}
        Security Onion incluye la pila Elastic, cuyos componentes son Logstash quien se encarga de recibir, procesar, normalizar y agregar los datos resultantes a la base de datos  Elasticsearch y Kibana que permite visualizar estos datos. El proceso comienza cuando Logstash recibe los datos sin procesar provenientes de múltiples fuentes, son normalizados por este componente y enviados a Elasticsearch para su almacenamiento. Kibana permite consultar la base de datos mediante una interfaz gráfica de usuario y utilizar esa información para propósitos de análisis de amenazas. \par
        ElastAlert (versión 0.1.39) es un framework que permite identificar y alertar sobre eventos anómalos o patrones de interés sobre los datos de Elasticsearch. También provee múltiples mecanismos para enviar alertas mediante distintas plataformas externas, tales como Slack \cite{slack}, correo electrónico, JIRA \cite{jira}, Telegram \cite{telegram} y muchos más. Tanto ElastAlert como los componentes de la pila Elastic están desplegados sobre contenedores Docker (versión 19.03.12) \cite{docker}. \par
        Es destacable que, aunque Security Onion cubre gran parte de los requerimientos de un SIEM, no posee los elementos que permiten completar un sistema de manejo y respuesta a incidentes; por esta razón y luego de una investigación sobre las alternativas posibles, se incluyó a TheHive (versión 3.3.0) \cite{thehive} y Cortex (3.0.1-1) \cite{thehive} como complemento de Security Onion. TheHive permite la gestión de incidentes de manera detallada y la colaboración con otros CSIRT mediante el uso compartido de información sobre incidentes en tiempo real; mientras que Cortex hace posible la automatización de las respuestas y operaciones ante incidentes utilizando los datos enviados por TheHive. \par

   \end{section}
   \begin{section}{Correlación de eventos: ElastAlert}
   \label{seccion4-5}
     A pesar de que Kibana permite consultar los datos almacenados en Elasticsearch y presentarlos de diversas maneras que resultan en una gran utilidad, carece de la capacidad de generar alertas cuando los datos coinciden con algún patrón, especialmente cuando estos datos son escritos y consultados en tiempo real en la base de datos. Con este objetivo, la plataforma integra a ElastAlert, siendo un componente confiable, modular y simple de configurar. Su funcionamiento se basa en dos componentes principales: reglas y alertas. Las primeras son utilizadas para comparar con los datos resultantes de las consultas que se hacen en forma constante a Elasticsearch, esta comparación consiste en hallar patrones o firmas definidas en las reglas dentro de los datos obtenidos de la consulta; si el resultado de la búsqueda es positivo, una alerta es disparada para notificar el evento. \par
     Las alertas consisten en mensajes que permiten notificar a otro sistema con el objetivo de que este último realice una acción sobre las causas del evento que detectó la regla o bien informar a los analistas y/o responsables definidos. En cualquiera de los dos casos, las alertas pueden incluir toda la información recabada en un formato definido, tales como plantillas o cualquier arreglo configurado a tal fin. \par
     Según la naturaleza de los eventos a clasificar, las reglas cuentan con un conjunto común de paradigmas de monitoreo, estos permiten identificar y generar alertas aprovechando las características de las anomalías al mismo tiempo que optimizan los recursos del resto del CSIRT en términos de hardware y atención de los analistas. Algunos de estos paradigmas se basan en el comportamiento, tales como la frecuencia que consiste en generar una alerta cuando se detectan N cantidad de eventos en un intervalo definido, el cambio de tasas de ocurrencia por arriba o abajo de un límite establecido como normal para un determinado tipo de eventos, cuando en los datos se encuentran presente campos que han sido previamente establecidos como parte de una lista blanca, negra u algún campo cuyo valor coincida con otros tipos de filtros, entre otros. Es posible definir y configurar tantas reglas como alertas sean necesarias. \par

   \end{section}
   \pagebreak
   \begin{section}{El panel de control general: TheHive }
     Como se mencionó en las secciones anteriores, Security Onion requiere de otros elementos capaces de realizar la gestión integral de incidentes y sus respuestas, elementos que sean capaces de condensar y presentar información a los analistas del CSIRT encargados de monitorear y responder a las anomalías e incidentes detectados.  TheHive es la herramienta que se eligió para esta tarea ya que es una plataforma de respuesta a incidentes de seguridad gratuita y de código abierto, cumpliendo así con uno de los requerimientos no funcionales del proyecto, referido al tipo de licencia y accesibilidad al código. Otra de las razones para la elección de esta plataforma en particular ha sido su escalabilidad y su integración con MISP, lo que permite compartir información sobre las amenazas detectadas con otros CSIRT de organizaciones aliadas. Las tres capacidades centrales son la elaboración de casos, la respuesta a estos y la anteriormente mencionada colaboración con otros CSIRT.\par
     \begin{figure}[H]
        \centering
        \includegraphics[width=1\textwidth]{./descripcion_sonion_imagenes/figura_23_alerta_panel_thehive.png}
        \caption{ Alertas recibidas en el panel de TheHive\cite{thehive}}
        \label{fig:alerta_panel_thehive}
     \end{figure}
        \FloatBarrier
        En cuanto a la elaboración de casos y tareas asociadas, estas se crean en base a las alertas recibidas \ref{fig:figura_21_arq_alm_elasticsearch}, donde el primer paso consiste en la creación de un caso para luego asociar este a una o varias de las alertas presentes utilizando la plantilla disponible \ref{fig:captura_kibana}, posteriormente es posible agregar tareas asociadas al caso, las cuales se pueden asignar a distintos analistas; a continuación es posible sumar métricas y campos personalizados, reducir el tiempo de búsqueda y recopilación de datos así como automatizar algunas tareas de recopilación de antecedentes en el manejo de incidentes mediante el uso del tablero, tal como se observa en la Figura \ref{fig:alerta_panel_thehive}. En el proceso de creación del caso, TheHive permite agregar cualquier otra información que se considere relevante, como etiquetas, archivos sospechosos de contener malware, etc a modo de evidencias.
        \begin{figure}[H]
        \centering
        \includegraphics[width=1\textwidth]{./descripcion_sonion_imagenes/figura_24_plantilla_creacion_casos.png}
        \caption{Plantilla para la creación de nuevos casos}
        \label{fig:figura_24_plantilla_nuevos_casos}
     \end{figure}
     \FloatBarrier
     En las figuras \ref{fig:figura_24_plantilla_nuevos_casos} y \ref{fig:dashboard_dinamico} se observan la plantilla de creación de nuevos casos y el tablero que listan los casos existentes, respectivamente.
     \begin{figure}[H]
        \centering
        \includegraphics[width=1\textwidth]{./descripcion_sonion_imagenes/figura_25_dashboard_dinamico_thehive.png}
        \caption{Lista de casos de TheHive}
        \label{fig:dashboard_dinamico}
     \end{figure}
     \FloatBarrier
     Luego de la creación de un caso, es posible sumarle todos los “observables” que sean necesarios, donde los observables son todos aquellos campos que se pueden agregar de forma manual y que constituyen fuentes de información para analizar cada caso. Una vez configurado un caso, estos son examinados por scripts llamados “analyzers” que correlacionan y filtran los datos del caso contra los provistos por otras instancias MIPS u otras fuentes de información como la propia base de datos local, servicios de resolución DNS, plataformas como Shodan\cite{shodan}, VirusTotal\cite{virustotal}, Google Cloud Visión\cite{vision-AI}, entre muchas otras. Los observables también se pueden obtener por datos de las alertas recibidas, los cuales son previamente configurados en ElastAlert. Como se mencionó anteriormente ElastAlert realiza consultas a Elasticsearch y con los resultados busca patrones de interés para realizar una notificación, obtenida esta última extrae datos que se consideran de interés para ser enviados a The Hive. Un ejemplo de esto puede ser un número de IP, tipo de protocolo, fecha que se generó el log, puerto de origen y/o destino. La alerta que llega a The Hive contiene todos estos datos, considerados observables. \par
     En la Figura \ref{fig:caso_de_uso_gestion_usuario_conf_thehive} se puede observar los distintos casos de uso para la gestión de usuario que tiene TheHive. Se observa que el administrador puede gestionar usuarios, lo que implica crearlos, bloquearlos o generar API KEYs para ellos. También es posible crear o modificar plantillas de casos y reportes, crear nuevos casos y métricas para estos. Así mismo, existe la posibilidad de  crear Observables, alertas y gestionar estas últimas: TheHive permite ver las alertas recibidas, ejecutar responders y agregar una alerta a un caso existente.\par

     \begin{figure}[H]
        \centering
        \includegraphics[width=1\textwidth]{./descripcion_sonion_imagenes/figura_28_thehive_user_conf.png}
        \caption{Casos de uso de gestión de usuario y configuraciones de TheHive}
        \label{fig:caso_de_uso_gestion_usuario_conf_thehive}
     \end{figure}
     \FloatBarrier
     En la figura \ref{fig:caso_de_uso_alertas_casos} se muestran las distintas opciones que se le presentan a un usuario de TheHive. Se aprecia que un usuario es capaz de visualizar alertas recibidas, agregar alertas a un caso existente y ejecutar responders de ser necesario. En cuanto a la gestión de casos, un usuario puede ver los casos existentes, crear uno nuevo, eliminar uno existente o unirlo con otro caso, cerrar un caso o en última instancia asignar este último a un usuario diferente. 
     \begin{figure}[H]
        \centering
        \includegraphics[width=1\textwidth]{./descripcion_sonion_imagenes/figura_29_thehive_alertas_casos.png}
        \caption{Diagrama de casos de uso de alertas y casos}
        \label{fig:caso_de_uso_alertas_casos}
     \end{figure}
     \pagebreak
     \begin{subsection}{Cortex}
        Luego de que el caso fue creado o sobre la misma alerta, el analista puede dar curso a una respuesta mediante “responders” que son scripts en los cuales se encuentra la respuesta del CSIRT a la amenaza. Tanto los responders como los analyzers se encuentran bajo la responsabilidad de Cortex, el subsistema encargado de procesar los casos de TheHive. Al final de esta sección, se presentan los diagramas de casos de uso correspondientes a Cortex.
        En la Figura \ref{fig:analizers_disponibles} se ven los analyzers disponibles. Estos son utilizados por los analistas para contrastar la información disponible con motores de búsqueda especializados como Shodan y VirusTotal. Los responders disponibles se observan en la Figura \ref{fig:ejemplos_responders_cortex}. Los analistas pueden activar o desactivar los mismos desde este panel.
     \begin{figure}[H]
        \centering
        \includegraphics[width=1\textwidth]{./descripcion_sonion_imagenes/figura_26_analyzers_disponibles.png}
        \caption{Algunos de los analyzers disponibles en Cortex}
        \label{fig:analizers_disponibles}
     \end{figure}
     \begin{figure}[H]
        \centering
        \includegraphics[width=1\textwidth]{./descripcion_sonion_imagenes/figura_27_responders_cortex.png}
        \caption{Ejemplos de responders utilizables en Cortex}
        \label{fig:ejemplos_responders_cortex}
     \end{figure}  
     \FloatBarrier
     En la Figura \ref{fig:caso_de_uso_gestion_usuarios_configuraciones_cortex} se observa el caso de uso de la gestión de usuarios y configuraciones para Cortex. Se muestran las capacidades de gestión tales como crear o bloquear usuarios, crear API KEYs para estos, crear organizaciones y agregarles usuarios a estas. Es posible asignar roles a los usuarios.
     %DIAGRAMA DE CASOS DE USOS DE CORTEX
     \begin{figure}[H]
        \centering
        \includegraphics[width=1\textwidth]{./descripcion_sonion_imagenes/figura_30_cortex_user_conf.png}
        \caption{Casos de uso de gestión de usuario y configuraciones de Cortex}
        \label{fig:caso_de_uso_gestion_usuarios_configuraciones_cortex}
     \end{figure}
     \FloatBarrier
     En la Figura \ref{fig:caso_de_uso_tablero_cortex} se observa los casos de usos del panel de Cortex. Las primeras burbujas presentan las pestañas principales del menú de Cortex. El administrador puede crear usuarios de una organización, asignarles roles y permisos, configurar analyzers y responders para una organización.
     \begin{figure}[H]
      \centering
      \includegraphics[width=1\textwidth]{./descripcion_sonion_imagenes/figura_31_cortex_tablero.png}
        \caption{Casos de uso del tablero de Cortex}
        \label{fig:caso_de_uso_tablero_cortex}
     \end{figure}
     \end{subsection}
   \end{section}

   \begin{section}{Integración con los sistemas de detección}
        En secciones anteriores se mencionó que Security Onion cuenta con NIDS y HIDS. Durante la configuración inicial del sistema se pueden especificar los NIDS a utilizar, para una configuración rápida de los sensores. Esto permite realizar una primera integración con el hardware disponible. Security Onion cuenta con OSSEC como HIDS para su autodefensa. La implementación de estos sistemas en el SIEM se llevo adelante en otro Proyecto Integrador.
   \end{section}
   \pagebreak
   \begin{subsection}{Suricata, Snort y Ossec}
        Suricata y Snort son motores de detección de amenazas en el tráfico de red. Ambos NIDS se basan en firmas o reglas para realizar la detección de amenazas. Estas firmas son actualizadas constantemente conforme a la aparición de nuevos tipos de ataques, exploits y malware. Si bien estos NIDS son gratuitos y de código abierto, Snort ofrece una versión paga, la cual cuenta con soporte para descargar las firmas actualizadas a la fecha. Por defecto Snort cuenta con las reglas básicas para la detección de amenazas bien conocidas. \par
    	Suricata, por otro lado, es desarrollado y mantenido por los colaboradores de la OISF, los cuales también dan soporte a las firmas ya que se actualizan las existentes y se agregan nuevas en forma permanente. Estas actualizaciones en las reglas son descargadas periódicamente mediante PulledPork, una utilidad que también es usada por Snort cuyo fin es descargar reglas y firmas desde distintos centros de investigación reconocidos en todo el mundo, como el SANS institute, Emerging Threats, entre otros. \par
        En el Cuadro \ref{table:4} se muestran las diferencias entre Snort y Suricata.
        \begin{table}[H]
            \centering
            \begin{tabular}{|m{10em}|m{11em}|m{11em}|}
                \hline 
               \multicolumn{1}{|c|}{Caracteristicas} & \multicolumn{1}{c|}{Snort} & \multicolumn{1}{c|}{Suricata}               \\ \hline % la primera fila se encuentra centrada
                    Desarrollador & CISCO & Open Information Security Foundation (OISF)  \\ 
                \hline
                    Lanzamiento  & 1998 & 2009 \\ 
                \hline
                    Lenguaje del código & C  & C  \\
                \hline
                    Sistema operativo & Linux, Windows y Mac OS X  & Linux, Windows y Mac OS X  \\
                \hline
                    Hilos & Monohilo  & Soporte múltiples hilos  \\
                \hline
                    Soporte IPv6 & Si  & Si  \\
                \hline
                    Reglas de Snort & Si  & Si \\
                \hline
                    Reglas de Emerging Threats & Si  & Si \\
                \hline
                    Formato de logs & unified2  & unified2   \\
                \hline
                    Compatible con Aanval & Si  & Si   \\
                \hline %linea final de tabla
            \end{tabular}
            \caption{Comparación entre Snort y Suricata}
            \label{table:4}
        \end{table}
        \FloatBarrier
        Por otro lado está OSSEC (reemplazado por  Wazuh en las versiones más recientes de Security Onion), es un IDS orientado a hosts (HIDS). Al igual que los NIDS anteriores está basado en firmas para la detección de amenazas, es gratuito y de código abierto. Las reglas pueden descargarse del repositorio disponible en github. \par
   \end{subsection}
   \begin{section}{Corolario}
        En este capitulo se mostró la solución elegida, sus distintas arquitecturas, sus componentes internos y los subsistemas asociados. Se pudo apreciar a TheHive como una solución complementaria a Security Onion que contribuyo a mejorar la gestión de incidentes.
        En el capítulo siguiente se desarrollo la configuración de un ambiente de prueba y se desplegaron las soluciones mencionadas anteriormente en distintas topologías. Esto permitió dar cumplimiento a los requerimientos funcionales 1 y 3. \par
   \end{section}
    \chapter{\Large Iteración I: “Despliegue e instalación de Security Onion en un ambiente de prueba”}
    En este capítulo describimos la instalación de Security Onion en la topología de red de prueba. Se instalará un nodo \textit{Master}, dos nodos \textit{Forward} y un nodo de \textit{TheHive}. \par
    En este proyecto se trabajó primero sobre un ambiente de prueba y después de producción. En ambos casos se utilizó un servidor central y un sistema operativo de virtualización sobre el que se crearon un conjunto de máquinas virtuales, cada una alojando un servidor con nodos \textit{Forward}, \textit{Master} y el correspondiente a TheHive - Cortex. Se utilizó de guía los componentes, el software y la arquitectura de conexión entre ellos, mencionados en la Descripción de Security Onion. \par
    \begin{section}{Topología de la organización}
    El paso inicial consistió en relevar la topología de la organización, para ello se dispuso de información inicial que fue brindada por los responsables del sector de redes. Posteriormente se realizaron diagramas topológicos con el fin de evaluar la situación y contemplar distintas estrategias para el despliegue de este proyecto.  
    En la Figura \ref{fig:iter1_top_unc} se observa un diagrama que representa la topología de la organización.\par
    Se puede apreciar que las dependencias están conectadas (mediante sus \textit{routers} de borde) entre sí y al mundo exterior mediante el Switch 1 de capa 3 que se encuentra en el \textit{datacenter}. Se observan los servidores 1, 2 y N que representan los activos de información de la organización, éstos están conectados mediante el \textit{Switch} 2 al resto de la organización. Existen M \textit{switches} para otras redes internas del \textit{datacenter}, todos conectados al \textit{Switch} 2. Este \textit{switch}, por lo tanto, conecta las redes internas del \textit{datacenter} y los activos de información con el \textit{Switch} 3, siendo este último el \textit{switch} troncal de la organización. Los enlaces anteriormente mencionados, incluidos los que conectan los \textit{routers} de borde de las dependencias y el \textit{Switch} 3, poseen un ancho de banda de 1 Gbps.\par
    Se observan otros componentes de la infraestructura de la organización, encargados de la conexión exterior con el resto del mundo. Estos son el \textit{Firewall} entre el \textit{Switch} 1 y el \textit{Switch} 5, los \textit{Routers} de borde 1 y 2 junto al \textit{Switch} 6 que se conecta a los proveedores ISP. Todos los enlaces entre estos componentes tienen como característica común un ancho de banda de 10 Gbps. \par
    \begin{figure}[H]
    \centering
    \includegraphics[width=1\textwidth]{./iteracion_1_imagenes/figura_topologia_UNC.png}
    \caption{Topología de la Organización}
    \label{fig:iter1_top_unc}
    \end{figure}
    \FloatBarrier
    \end{section}
    \begin{subsection}{Definición y configuración de las redes a observar}
    Se analizó el ancho de banda de las dependencias existentes y se decidió monitorear dos de ellas. Se seleccionaron las de mayor volumen de tráfico en función de mediciones realizadas a lo largo de una semana y de datos históricos. Las dependencias seleccionadas tenían un enlace con ancho de banda de 1 Gbps cada una. \par
    Las Figuras \ref{fig:figura_35_trafico_dia} muestra el volumen de tráfico medido en un periodo de tiempo de un día (exceptuando las horas en las que la actividad era mínima), mientras que en la Figura \ref{fig:figura_36_trafico_semana} muestra un periodo de una semana. Estos gráficos fueron obtenidos mediante el uso de LibreNMS \cite{librenms} (versión 1.48).\par
    Los datos de las Figuras 5.2 y 5.3 fueron tomados desde la interfaz Gig0 del \textit{Switch} 1 de la Figura 5.1, esto quiere decir que el volumen de tráfico corresponde a la Dependencia1. Las interfaces Gig3, Gig-M y Gig-J del \textit{Switch} 1 tenían un comportamiento similar.\par
    \begin{figure}[H]
    \centering
    \includegraphics[width=1\textwidth]{./iteracion_1_imagenes/figura_35_trafico_dia.png}
    \caption{Tráfico correspondiente a una dependencia, medido durante un día }
    \label{fig:figura_35_trafico_dia}
    \end{figure}
    \begin{figure}[H]
    \centering
    \includegraphics[width=1\textwidth]{./iteracion_1_imagenes/figura_36_trafico_semana_editada.png}
    \caption{Tráfico medido durante el periodo correspondiente a una semana}
    \label{fig:figura_36_trafico_semana}
    \end{figure}
    \FloatBarrier
    En la Figura 5.3 se observa el tráfico de la interfaz Gig0 en el periodo de una semana, donde la velocidad promedio correspondiente a la entrada es de 11,95 Mbps, mientras que de salida es de 47,24 Mbps. Se observaron picos de 300 Mbps de tráfico.\par
    Como la medición corresponde a la interfaz Gig0, el tráfico entrante a esta corresponde al de subida de la Dependencia 1, mientras que el tráfico saliente corresponde al de bajada de esta dependencia.
    \end{subsection}
    \begin{subsection}{Selección de hardware}
    \label{seleccion_hw}
    En primer lugar se examinaron los requisitos de hardware mínimos y recomendados por cada uno de los fabricantes. Para esto se tuvo en cuenta el ancho de banda y el tráfico de los enlaces a monitorear, descritos en la sección precedente. Los requerimientos se presentan a continuación, en los Cuadros \ref{table:15} y \ref{table:5} (para nodos Monolítico y Distribuido, respectivamente). \par
    En base a la Figura \ref{fig:iter1_top_unc}, los datos referidos al tráfico y el ancho de banda de los enlaces a monitorear, se dimensionó el hardware necesario para el servidor central, teniendo en cuenta los requerimientos del fabricante y las limitaciones disponibles en la organización.\par
    %Tabla HW nodo Monolitico
    \begin{table}[H]
    \centering
    \begin{tabular}{|m{10em}|m{10em}|}
    \hline 
    Requerimiento  & Nodo Standalone \\ 
    \hline
    Cantidad de CPU - Arquitectura &  10 nucleos vCPU - x86-64  \\ \hline
    Memoria RAM  &  32 GB  \\ 
    \hline
    Almacenamiento necesario   & 200 GB  \\
    \hline
    Cantidad de interfaces de red  & 2 (administración y monitoreo) \\
    \hline %linea final de tabla
    \end{tabular}
    \caption{Requerimientos de hardware para un nodo Standalone, según el fabricante.}
    \label{table:15}
    \end{table}
     %Tabla HW topología distribuida  
    \begin{table}[H]
    \centering
    \begin{tabular}{|m{9em}|m{9em}|m{9em}|m{9em}|}
    \hline 
    Requerimiento  & Nodo \textit{Master} &  Nodo \textit{Forward} & TheHive y Cortex \\ 
    \hline
    Cantidad de CPU - Arquitectura & Mínimo de 8 núcleos vCPU - x86-64 & Mínimo de 12 núcleos vCPU - x86-64 & Mínimo de 8 núcleos vCPU - x86-64 \\ 
    \hline
    Memoria RAM  & 12 a 128 GB & 128 a 256 GB & A partir de 8 GB \\ 
    \hline
    Almacenamiento necesario & Mínimo de 1 TB  & Mínimo de 540 GB & A partir de 60 GB \\
    \hline
    Cantidad de interfaces de red & 1 (administración) & 2 (administración y monitoreo) & 1 (administración) \\
    \hline %linea final de tabla
    \end{tabular}
    \caption{Requerimientos de hardware recomendados por el fabricante para el monitoreo de un enlace de 1 Gbps}
    \label{table:5}
    \end{table}
    Debido a las restricciones en la disponibilidad del hardware, se realizó una implementación aproximada, teniendo en cuenta los recursos disponibles en la organización. A continuación se muestran los recursos de hardware utilizados, según se trate del despliegue de una topología Monolítica (Cuadro \ref{table:16}) o de una Distribuida (Cuadro \ref{table:12}). \par
    \begin{table}[H]
    \centering
    \begin{tabular}{|m{10em}|m{10em}|}
    \hline 
    Requerimiento  & Nodo Standalone \\ 
    \hline
    Cantidad de CPU - Arquitectura &  4 núcleos vCPU - x86-64  \\ 
    \hline
    Memoria RAM  &  16 GB  \\ 
    \hline
    Almacenamiento necesario   & 500 GB  \\
    \hline
    Cantidad de interfaces de red  & 2 (administración y monitoreo) \\
    \hline %linea final de tabla
    \end{tabular}
    \caption{Requerimientos de hardware utilizados para el despliegue del nodo Standalone.}
    \label{table:16}
   \end{table}
    %En el caso de la topología Distribuida, el hardware que se utilizó para cada nodo se muestra en el Cuadro \ref{table:12}.
    \begin{table}[H]
    \centering
    \begin{tabular}{|m{9em}|m{9em}|m{9em}|m{9em}|}
    \hline 
    Requerimiento  & Nodo \textit{Master} &  Nodo \textit{Forward} & TheHive y Cortex \\ 
    \hline
     Cantidad de CPU - Arquitectura & 8 nucleos vCPU - x86-64 & 10 nucleos vCPU - x86-64 & 8 nucleos vCPU - x86-64 \\ 
    \hline
    Memoria RAM  & 16 GB & 32 GB & 8 GB \\ 
    \hline
    Almacenamiento necesario & 500 GB  & 200 GB & 60 GB \\
    \hline
    Cantidad de interfaces de red & 1 (administración) & 2 (administración y monitoreo) & 1 (administración) \\
    \hline %linea final de tabla
    \end{tabular}
    \caption{Requerimientos de hardware utilizado según el tipo de nodo desplegado.}
    \label{table:12}
    \end{table}
    \end{subsection}
    \pagebreak
    \begin{subsubsection}{Características del hardware empleado}
    Se dispuso de varios servidores centrales en los cuales se albergaron las máquinas virtuales de este proyecto. Si bien cada uno de ellos contaba con distinta capacidad de hardware, estas diferencias eran sólo cuantitativas, ya que las características del hardware para todos los servidores eran las mismas. A continuación, se detallan las más relevantes:
   \begin{itemize}
    \item En cuanto a los núcleos de CPU virtuales, su frecuencia era de 2.4 Ghz, basados en procesadores físicos Intel Xeon E5620.
    \item Las memorias RAM pertenecían a la tecnología DDR3 y su frecuencia de refresco era de 2400 MHz.
    \item Los discos utilizados fueron del tipo mecánico con una velocidad de transferencia para lectura de 196 MB/s y para escritura de 154 MB/s, con interfaz SATA III y 7200 rpm de velocidad de rotación.
    \end{itemize}
    \end{subsubsection}
    
    \begin{subsection}{Pérdidas de paquetes}
    Como consecuencia de no satisfacer los requerimientos de hardware recomendados por el fabricante, para los nodos que contienen sistemas IDS, se detectó una pérdida de paquetes que varía según el volumen de tráfico. El estudio de este problema fue realizado por otro grupo, que llevó a cabo su Proyecto Integrador sobre sistemas IDS. Se realizaron pruebas con diferentes velocidades de tráfico y se observo un porcentaje de paquetes que se perdía dependiendo de la velocidad del enlace, como se aprecia en el Cuadro \ref{table_13}.\par
    \begin{table}[H]
    %\centering
    \resizebox{\textwidth}{!}{
    \begin{tabular}{|c|c|c|} 
    \hline
    Velocidad (Mbps) &  Pérdidas de paquetes (\%) & Alertas sin detectar (\%)\\
    \hline
    800 & 25 & 40 \\
    \hline
    400 & 12 & 20 \\
    \hline
    200 & 0 & 0 \\
    \hline %linea final de la tabla
    \end{tabular}
    }
    \caption{Perdida de paquetes y alertas no detectadas, según la velocidad del enlace}
    \label{table_13}
    \end{table}
    La pérdida de paquetes por parte de los IDS tuvo como consecuencia una disminución en la cantidad de alertas detectadas. En base a pruebas llevadas a cabo con TCPreplay\cite{tcpreplay} enviando PCAPs con ataques a los sistemas IDS, con duración de un día cada prueba, se estimó el porcentaje de alertas sin detectar según el porcentaje de pérdidas de paquetes. Los resultados de estas pruebas se observan en el Cuadro \ref{table_13}.
    \end{subsection}
    \begin{section}{Monitoreo de red}
    A pesar de conocer, por la sección anterior, que el rendimiento de una topología monolítica no es óptimo en entornos de alto rendimiento, se decidió desplegar esta topología en primer lugar para familiarizarnos con la solución.\par
    \end{section}
    \begin{subsection}{Topología Monolítica de Monitoreo}
    \label{subsec-topo-mono}
    En la Figura \ref{fig:figura_33_a}, se observa una topología de prueba desarrollada en una red aislada, a modo de experiencia de laboratorio, en el cual se desplegó la solución con una arquitectura monolítica.\par
    El experimento consistió en demostrar la capacidad de monitoreo y detección de incidentes de Security Onion, mediante el despliegue de un nodo Monolítico y tres terminales de computadoras, una de las cuales (C) simuló ser el atacante.\par
    En primer lugar, se conectaron las terminales “A” y “B” (la del analista y el generador de logs, respectivamente) a un switch al cual se conectó, mediante su interfaz “Eth1”, uno de los enlaces del servidor de Security Onion. Este enlace, llamado “Eth0” en la Figura \ref{fig:figura_33_a}, sirvió a fines de que el analista en el terminal “A” simule consultas y visualice datos en Security Onion, mientras que el terminal “B” envíaba logs mediante Filebeat \cite{filebeat}, conteniendo información contextual de los activos de la red. \par
    \begin{figure}[H]
    \centering
    \includegraphics[width=0.8\textwidth]{./iteracion_1_imagenes/figura_33_a_topologia_de_prueba_1.png}
    \caption{Topología monolítica de laboratorio}
    \label{fig:figura_33_a}
    \end{figure}
    El tercer terminal (C) se conectó mediante un enlace a la interfaz “Eth1” del nodo Monolítico, enviando tráfico simulado mediante PCAPS conteniendo acciones ofensivas. Para lograr esta acción, en el terminal C se utilizó TCPreplay \cite{tcpreplay} para reenviar los PCAPS mencionados. Esta interfaz estaba configurada para monitorear el tráfico en busca de paquetes que pudieran contener una acción ofensiva. 
    Posteriormente se colocó en producción un nodo Monolítico. En la Figura \ref{fig:iter1_top_m_unc} se observa la topología de la organización incluyendo la conexión con este nodo. Este tiene su interfaz de monitoreo conectada al Switch 3, donde recibe el tráfico reenviado que se produce entre la Dependencia 1 y el mencionado switch.
    \begin{figure}[H]
    \centering
    \includegraphics[width=1\textwidth]{./iteracion_1_imagenes/figura_topologia_m_unc.png}
    \caption{Despliegue de una topología monolítica de monitoreo}
    \label{fig:iter1_top_m_unc}
    \end{figure}
    \FloatBarrier
    Se realizaron pruebas de los requerimientos funcionales 1 y 3: se comprobó el funcionamiento de la recolección y almacenamiento de datos de incidentes de seguridad, al recibir y analizar tráfico proveniente de la Dependencia 1. \par
    %El nodo monolítico de Security Onion dispone de dos conexiones. La correspondiente al puerto eth0 es la que permite la administración del sistema y la consulta de sus datos, mientras que la interfaz eth1 es la destina a monitorear el tráfico de una red. En nuestro experimento con esta arquitectura, todo el tráfico fue simulado usando los PCAPS mencionados anteriormente. \par
    %En la Figura \ref{fig:iter1_top_m_unc} se observa la topología de la organización con un nodo Monolítico. Este tiene su interfaz de monitoreo conectada al Switch 3, donde recibe el tráfico reenviado que se produce entre la Dependencia 1 y el mencionado switch.
    
    \end{subsection}
    
    \begin{subsection}{Topología Distribuida de Monitoreo}
    \label{sec_topo_dist}
    Después de comprobar el funcionamiento de los principales componentes de Security Onion y obtener la experiencia descrita en la sección anterior, se procedió a considerar el despliegue de una topología distribuida. Este tipo de topología presenta considerables ventajas respecto de su alternativa monolítica, fundamentalmente en términos de rendimiento del hardware y flexibilidad de adaptación, como se mencionó en el Capítulo~\ref{tipos_de_arquitectura}.\par
    En primer lugar, se desplegó una versión simplificada para verificar el funcionamiento de los componentes de los dos nodos que la componen (nodos \textit{Forward} y \textit{Master}) y la comunicación entre ellos.
    \begin{figure}[H]
    \centering
    \includegraphics[width=1\textwidth]{./iteracion_1_imagenes/figura_33_b_topologia_de_prueba_2.png}
    \caption{Versión elemental de una topología distribuida}
    \label{fig:topologia_distribuida_1}
    \end{figure}
    Se observa en la Figura \ref{fig:topologia_distribuida_1} los componentes fundamentales de Security Onion para poder llevar a cabo esta topología: un nodo \textit{Forward} conectado a la interfaz a monitorear y un nodo \textit{Master} conectado al switch de la red interna. El nodo \textit{Master} y el \textit{Forward} se comunican mediante un enlace de administración (interfaz eth0 del nodo \textit{Forward}), que es el que está conectado al Switch. \par
    En este caso, como en el descrito en la sección anterior (\ref{subsec-topo-mono}), se utilizaron logs y tráficos de red obtenidos con anterioridad. El objetivo de esta experiencia fue comprobar el correcto funcionamiento de los dos nodos. Adicionalmente, con el motivo de probar la generación de notificaciones de alertas, se comprobó el funcionamiento de ElastAlert, que forma parte de los componentes del nodo \textit{Master}. \par
    La experiencia consistió en enviar logs desde la terminal con Filebeat hacia el nodo \textit{Master}, para que posteriormente fueran filtrados. Para esta tarea se empleó Logstash junto a un \textit{plugin} llamado Grok. Se buscó identificar campos de interés que estuvieran presentes en los \textit{logs} y se hallaron direcciones IP. Se crearon alertas en ElastAlert que se activaron al detectar estas IP y se enviaron mensajes por un servidor de correo electrónico (propio de la organización) y aplicaciones (Telegram y Slack). Los resultados de este experimento verificaron el RF4, que se trata en la iteración II (Capítulo \ref{iteracion2}). \par
    Si bien Security Onion incluye gran parte de los componentes de un SIEM, es necesario un sistema adicional que sea capaz de recolectar las alertas generadas en primera instancia por el nodo \textit{Master} y manipular esta información para lograr un manejo eficaz de los incidentes. Este sistema es TheHive. \par
    TheHive recibe las alertas generadas en el nodo \textit{Master} por ElastAlert y las introduce en un proceso de correlación de incidentes para proporcionar mayor información del mismo a los analistas del CSIRT. Cortex es un componente asociado a TheHive, que permite desarrollar y automatizar respuestas a incidentes. TheHive y Cortex, junto a los componentes del nodo \textit{Master} (Kibana y Squert) son los únicos componentes visibles con los que interactúan los analistas. \par
    
    Luego de la incorporación de TheHive, se volvió a repetir el experimento detallado anteriormente (Figura \ref{fig:topologia_distribuida_1}), esta vez con el objetivo de que TheHive recibiera las alertas generadas en el nodo \textit{Master} mediante ElastAlert. La topología del experimento se observa en la Figura \ref{fig:topologia_distribuida_2}.\par
    \begin{figure}[H]
    \centering         \includegraphics[width=1\textwidth]{./iteracion_1_imagenes/figura_33_e_topologia_de_prueba_3.png}
    \caption{Experiencia de laboratorio de la topología distribuida con todos sus componentes}
    \label{fig:topologia_distribuida_2}
    \end{figure}
    \FloatBarrier
    De esta manera, se completaron los componentes del SIEM al ofrecer un manejo centralizado de los datos de incidentes de seguridad de la información. Cuando el nodo \textit{Forward} identificó un  ataque, notificó al nodo \textit{Master} del mismo. En este último, ElastAlert generó alertas que fueron enviadas al servidor de TheHive, quien las presentó a los analistas junto a información contextual que intentó correlacionar con información presente en su base de datos.\par
    
    Finalmente, se procedió a desplegar la configuración final de la topología de la solución en este proyecto. Esta consistió en agregar un segundo nodo \textit{Forward} para monitorear el tráfico de una dependencia adicional, como se muestra en la Figura \ref{fig:iter1_top_d_unc}.
    \begin{figure}[H]
    \centering
    \includegraphics[width=1\textwidth]{./iteracion_1_imagenes/figura_topologia_d_unc.png}
    \caption{Despliegue de una topología distribuida de monitoreo}
    \label{fig:iter1_top_d_unc}
    \end{figure}
    La Figura \ref{fig:iter1_top_d_unc} muestra la topología de prueba donde se instaló Security Onion. Se observan los proveedores de conexión a internet (ISP) y la conexión con el \textit{router} de borde de la organización. Este último se conecta mediante un enlace \textit{gigabit} en el puerto Gig0 al puerto Gig4 del \textit{Switch} 1 (capa 3). \par
    El \textit{Switch} 1 es un dispositivo de red de capa 3, que conecta las dependencias entre sí y con el \textit{datacenter}. Las conexiones mencionadas se implementan físicamente sobre un enlace \textit{gigabit} de fibra óptica, pero virtualmente sobre redes tipo VLAN. De esta manera, por ejemplo, la Dependencia 1 está conectada físicamente por un enlace \textit{gigabit} a través de su puerto Gig1 de su \textit{router} de borde, con el puerto Gig5 del Switch 1 capa 3. Desde el punto de vista lógico, este enlace pertenece a la VLAN 1 de la organización. La situación descrita es análoga para el resto de las dependencias de la Universidad. \par
    Por otro lado, el \textit{Switch} 1 (en adelante SW 1), está conectado a los nodos \textit{Forward} 1 y 2 de Security Onion. Como resultado, es posible reenviar el tráfico entre SW1 y las dependencias 1 y 2 hacia los puertos Gig0 y Gig2 de SW1 que conectan SW1 con los nodos \textit{Forward} 1 y 2. \par
    SW1 está conectado con el \textit{Switch} 2 perteneciente al \textit{datacenter}, al cual se conectan a su vez los terminales de los analistas, los nodos \textit{Master} y \textit{Forward} (1 y 2) de Security Onion y TheHive. Esta topología permite desplegar una arquitectura distribuida de Security Onion, que fue la elegida para este proyecto.
    \end{subsection}
    \begin{section}{Verificación de RF1 y RF3}
    Luego de haber desplegado la topología distribuida que se observa de la Figura \ref{fig:iter1_ver_RF1_RF2}, se verificó el cumplimiento de los requerimientos RF1 y RF3. Para ello se llevó a cabo una acción ofensiva sobre un servidor ubicado en el \textit{datacenter}. Se seleccionó este servidor mediante un acuerdo con los responsables del área, para fines de evaluación de este proyecto. Este activo de la organización fue configurado a modo de prueba con un servidor web Apache versión 2.4.46, que contaba con una formulario web en PHP versión  7.2.21  y una base de datos MySQL versión 8.0.17.\par
    \begin{figure}[H]
    \centering
    \includegraphics[width=1\textwidth]{./iteracion_1_imagenes/Topologia de despliegue descentralizada RF2, RF6 y RF4.png}
    \caption{Despliegue de una topología distribuida de monitoreo}
    \label{fig:iter1_ver_RF1_RF2}
    \end{figure}
    La acción ofensiva consistió en un ataque de reconocimiento que partió desde el \textit{host} ATACANTE ubicado dentro de la Dependencia 1 hacia la víctima llamada Servidor 1 que se encontraba dentro del \textit{datacenter}.
    Para realizar el ataque de reconocimiento se utilizaron rutinas de NMAP \cite{nmap} (versión 7.80) configuradas a tal fin: \textit{nmap -T4 -A -v IP\_VICTIMA} . Los parámetros utilizados fueron:
    \begin{itemize}
    \item -T4: para un escaneo intensivo (disminuye el tiempo de ejecución entre \textit{scripts})
    \item -A: habilita la detección del sistema operativo de la víctima y su versión, los \textit{scripts} de escaneo y \textit{traceroute}.
    \item -v: habilita el modo “verboso”.
    \item IP\_VICTIMA: es la dirección IP objetivo de este reconocimiento.
    \end{itemize}
    En la Figura \ref{fig:squert-nmap} se observa el panel de visualización de eventos de Squert.  En el recuadro rojo se encuentra la detección del ataque de reconocimiento realizado. El resto de la información que se aprecia en el panel principal corresponde a otros eventos que el sistema estaba detectando. Los eventos se agrupan según la categoría a la que pertenecen y un indicador de colores (barra vertical a la derecha del contador de eventos de cada fila) indica su prioridad de atención. %*Por defecto, Squert tiene su propia categoría de prioridades.
    Luego se observan la cantidad de eventos según su dirección, hora del último incidente ocurrido junto con su ID, una descripción de la firma del evento y el porcentaje de ocurrencia respecto al total de incidentes detectados.
    \begin{figure}[H]
    \centering
    \includegraphics[width=1\textwidth]{./iteracion_1_imagenes/Squert_NMAP.png}
    \caption{Eventos visualizados en Squert}
    \label{fig:squert-nmap}
    \end{figure}
    \FloatBarrier
    En la Figura \ref{fig:thehive-nmap} se observa la visualización de la detección en TheHive. Si bien pueden ser similares, Squert se limita a mostrar los ataques similares agrupados en la misma categoría de incidente al que pertenecen, mientras que TheHive recibe las alertas de ElastAlert y presenta los eventos de manera individual. Esto permite a los analistas crear casos con cada incidente, correlacionarse con otros ataques, etc, lo que conduce a una gestión integral de eventos de seguridad.
    \begin{figure}[H]
    \centering
    \includegraphics[width=1\textwidth]{./iteracion_1_imagenes/TheHive-NMAPEditado.png}
    \caption{Detección en TheHive del ataque de reconocimiento}
    \label{fig:thehive-nmap}
    \end{figure}
    \FloatBarrier
    Se verificó que las direcciones IP reportadas en el incidente por Squert y TheHive, coincidieran con las de la víctima y el atacante. Además mientras se estaba ejecutando el ataque, se hizo un seguimiento mediante Wireshark \cite{wireshark} filtrando las direcciones IP de origen y destino. Esto permitió corroborar el flujo de datos entre ambos.
    Los ataques fueron detectados por los nodos \textit{Forward} de la Dependencia 1 y \textit{Forward} Activos. \par
    Con esta prueba se verifica el cumplimiento de los requerimientos funcionales 1 y 3: “recolectar y almacenar datos de incidentes de seguridad en la infraestructura de la red corporativa ” y “visualizar las alertas en un tablero de mando”.\par
    \end{section} 
    
    
    \begin{section}{Colorario}
    En cuanto a la topología de la organización (Figura \ref{fig:iter1_top_unc}), podemos concluir que, la mejor opción para monitorear el tráfico entre las dependencias de la organización y el exterior fue conectar los nodos \textit{Forward} al \textit{Switch} 1 (el \textit{switch} troncal), para recibir el tráfico reenviado entre las dependencias y el mencionado \textit{switch}. Por otro lado, con el objetivo de lograr una defensa de punto final sobre activos de información específicos, fue necesario conectar un nodo \textit{Forward} al \textit{switch} (\textit{Switch} 2) al que se encuentran conectados estos activos. Estas conexiones se vieron reflejadas en la Figura \ref{fig:iter1_ver_RF1_RF2}. \par
    Se observó que la topología Monolítica fue ineficiente para el uso en entornos de producción, dado que la pila Elastic y los componentes de los sensores IDS de este nodo demandaron un uso intensivo de hardware. Si bien este tipo de topología fue útil a fines de probar el sistema o para monitorear enlaces de reducido tráfico y poco ancho de banda, resultó ineficaz en entornos más complejos. Esto se debió a los enormes requisitos de hardware necesarios para monitorear un enlace de 1 Gbps y considerando que el despliegue final requirió el monitoreo de múltiples enlaces con este ancho de banda, este tipo de topología fue incapaz de escalar utilizando el hardware disponible.\par
    Como consecuencia de la incapacidad de escalamiento horizontal descrita y siendo esta un requerimiento no funcional de este proyecto, se desplegó una topología distribuida de Security Onion. Los resultados del monitoreo fueron positivos, por lo cual se dio cumplimiento al RNF3, en simultáneo con la verificación de RF1 y RF3.\par
    

    \end{section}
    \chapter{Iteración II: “Información contextual y envió de alertas de seguridad”}
    Con el objetivo de recibir y procesar información contextual de los activos que se ven afectados durante un incidente, fue necesario configurar el servidor Master de Security Onion con otros servicios que permitieran recibir información diferente de la provista por los sensores. Esta información refleja el estado de los servicios en un servidor, su conexión a la red, parámetros de uso de hardware como el nivel de ocupación de disco, temperatura, uso de la memoria RAM, los logs referidos al procesamiento de las peticiones que recibe un servidor, etc. \par
    Una anomalía en los datos de esta información no es suficiente para confirmar un ataque, pero es útil para indicar que algo puede estar ocurriendo. En el caso de un ataque confirmado, estos datos representan evidencia forense y sirven para enriquecer modelos de amenazas complejas. \par
    \begin{section}{Envío de logs con Filebeat}
    Filebeat es un servicio que permite enviar información a Logstash desde múltiples directorios. El proceso comienza cuando el servicio lee línea por línea los archivos de entrada y envía los logs de la misma manera a Logstash. Este los recibe en su puerto 5044 por defecto, posteriormente puede filtrarlos para separar los campos de los logs o bien almacenarlos sin filtrar. \par
    En nuestro caso particular, al principio enviamos los logs mediante Filebeat y no los procesamos. Esto generó que ElastAlert no pudiera identificar información clave para realizar una correlación, como la que se encuentra en los campos de puertos y direcciones IP de origen y destino, estampas de tiempo, tipo de peticiones, etc. \par
    Posteriormente utilizamos grok para filtrar los logs que estaba recibiendo Logstash antes de almacenarlos, de manera que la detección basada en correlaciones fue posible.\par
    Como se observa en la Figura \ref{fig:iter2_despl_dist} y tal como se describió en la sección \ref{subsection:topo_dist}, realizamos el envío mediante Filebeat de logs de peticiones que fueron parte ataques a activos de la organización en el pasado. 
    \begin{figure}[H]
    \centering
        \includegraphics[width=0.7\textwidth]{./iteracion_1_imagenes/figura_33_c_topologia_de_prueba_3.png}
        \caption{Despliegue distribuido de prueba}
        \label{fig:iter2_despl_dist}
    \end{figure}
    \FloatBarrier
    Como se mencionó anteriormente, en primer lugar los logs se almacenaron sin filtrar, lo que ocasionó la pérdida de la capacidad de correlación. En la Figura \ref{fig:iter2_logs_crudos} se puede observar el almacenamiento de estos registros.
    %INSERTAR ACA FIGURA 6.2 "LOGS SIN PROCESAR"
    \begin{figure}[H]
    \centering
        \includegraphics[width=1\textwidth]{./iteracion_2_imagenes/1_kibana_logs_1EDITADA.png}
        \caption{Almacenamiento de logs sin procesar}
        \label{fig:iter2_logs_crudos}
    \end{figure}
    \FloatBarrier
    Posteriormente se repitió la experiencia pero esta vez los logs que se recibian en Logstash eran procesados utilizando grok. Grok es un plugin de la pila Elastic que permite filtrar datos sin ningún tipo de estructura y generar información estructurada capaz de ser consultada.
    Los resultados se pueden ver en la Figura \ref{fig:iter2_logs_filtrados}.\par
    %INSERTAR ACA FIGURA 6.3 "LOGS PROCESADOS"
    \begin{figure}[H]
    \centering
        \includegraphics[width=1\textwidth]{./iteracion_2_imagenes/kibana_logs_parseados_2EDITADO.png}
        \caption{Almacenamiento de logs procesados por Logstash}
        \label{fig:iter2_logs_filtrados}
    \end{figure}
    Logstash posee varios filtros para todos los tipos de entradas que soporta. 
    Fue posible editar y administrar estos filtros utilizando archivos de configuración ubicados en la carpeta \textit{/etc/logstash/conf.d.available/}. Para este proyecto se eligió editar el filtro que procesa los datos provenientes de Filebeat, como se indicó anteriormente.\par
    El proceso de creación de este filtro consistio en copiar el archivo de configuración de Filebeat: \textit{9500\_output\_beats\_custom.conf} que se encuentra en el directorio mencionado anteriormente. El archivo fue copiado a la carpeta \textit{/etc/logstash/custom}, donde fue modificado para crear el filtro de grok. Este fue creado teniendo en cuenta el cuerpo de los mensajes provenientes del archivo de logs analizado, dado que se conocían los campos de interés que contenían los logs y que por lo tanto se deseaba extraer. Finalmente se reinicio el servicio de Logstash y se comprobó que el archivo \textit{logstash.log} para verificar que no hubiera errores.\par
    Es importante mencionar que la información resultante es enviada por Logstash a Elasticsearch, donde es consultada por ElastAlert. Este ultimo componente es el encargado de correlacionar los campos de los logs y enviar notificaciones.

    \end{section}
    
    \begin{section}{Envío de alertas de seguridad}
    El envío de alertas de seguridad y la notificación a los responsables de los activos que se ven afectados, fue realizado en este trabajo mediante ElastAlert. Este componente fue seleccionado ya que permite correlacionar la información presente en los campos de logs guardados en elasticsearch, así como enviar notificaciones cuando se produce la detección de incidentes. \par
    ElastAlert es un framework que detecta patrones en los datos consultados a Elasticsearch, como anomalías, picos, etc. Como se mencionó en la sección 4.5, basa su funcionamiento en dos componentes: reglas y alertas. \par
    Las alertas consisten en mensajes que permiten notificar a un usuario final o a otro sistema, de que un evento de seguridad ha ocurrido. En este proyecto se decidió que las notificaciones fueran enviadas por correo electrónico a los responsables de un área de la organización.\par
    Se escribieron archivos de configuración para cada tipo de evento, de manera tal que las reglas que disparan cada alerta lo hagan según el comportamiento y naturaleza del incidente. En las figuras \ref{fig:iter2_1_codigo} a \ref{fig:iter2_4_codigo} se observa el código desarrollado para el envío de alertas en ocasión de producirse un reconocimiento y escaneo de puertos. Las figuras corresponden al mismo archivo de configuración, pero cada una muestra una parte del código para facilitar la explicación del mismo.\par
    En la Figura \ref{fig:iter2_1_codigo} se muestra la primer parte del código de configuración de una regla:
    \begin{itemize}
        \item \textit{es\_host}: el nombre del host de la base de datos elasticsearch.
        \item \textit{es\_port}: el número de puerto del host de elasticsearch mediante el cual ElastAlert se contactara con la base de datos.
        \item \textit{name}: nombre de la regla. Debe ser único.
        \item \textit{type}: especifica el tipo de regla. En este caso es del tipo “frecuency”, que determina que esta regla se activará si al menos se produce cierto número de eventos en determinado tiempo.
        \item \textit{num\_events}: especifica la cantidad de eventos necesarios para activar este tipo de regla.
        \item \textit{timeframe}: determina el marco temporal necesario para este tipo de reglas. tiene un sub parámetro que especifica la unidad temporal y la cantidad de estas. En este caso optamos por un marco temporal de un (1) minuto.
    \end{itemize}
    \begin{figure}[H]
    \centering
        \includegraphics[width=0.7\textwidth]{./iteracion_2_imagenes/3-codigoAlerta-1.png}
        \caption{Primera parte del código de configuración de una regla}
        \label{fig:iter2_1_codigo}
    \end{figure}
    \FloatBarrier
    En la Figura \ref{fig:iter2_2_codigo} se muestran los campos implicados en la consulta a elasticsearch:
    \begin{itemize}
        \item \textit{index}: es el índice a consultar
        \item \textit{use\_strftime\_index}: es una variable booleana que, en el caso de ser verdadera, da formato al índice utilizando “\textit{datetime\_strftime}” para cada consulta. Este último método de Python crea un string representando el tiempo en un formato específico. Se utiliza en el caso de que la consulta realizada incluya varios días, con lo que los índices se concatenan utilizando “;” permitiendo una búsqueda más eficiente.
        \item \textit{filter}: determina los filtros de elasticsearch utilizados para la consulta. Utiliza los subparametros “term”  y “category” para indicar la clave a buscar. En este caso utilizamos “category: scan” para filtrar todos los eventos que pertenezcan a la categoría del reconocimiento de puertos.
        \item \textit{query\_term}: permite agrupar las consultas, en este caso agrupamos las alertas resultantes por direcciones IP de origen y destino, así como todos los eventos que tengan el campo “alert”.
        \item \textit{realert}: ignora las alertas repetidas en un cierto periodo de tiempo. En la práctica esto nos permitió evitar saturar con notificaciones a los responsables de los activos afectados, ya que el reconocimiento de puertos es un incidente extremadamente común y periodico, produciéndose muchas veces por día.
    \end{itemize}
    \begin{figure}[H]
    \centering
        \includegraphics[width=1\textwidth]{./iteracion_2_imagenes/4-codigoAlerta2.png}
        \caption{Segunda parte del código de configuración de una regla}
        \label{fig:iter2_2_codigo}
    \end{figure}
    \FloatBarrier
    En la Figura \ref{fig:iter2_3_codigo} se observan los campos utilizados para armar el cuerpo del mensaje y el asunto:
    \begin{itemize}
        \item \textit{doc\_type}: especifica el tipo de documento a consultar en elasticsearch.
        \item \textit{alert\_subject}: permite personalizar el mensaje de una alerta agregando un pequeño resumen. En este proyecto lo utilizamos en todas las reglas, para que el “asunto” del correo electrónico contenga un mensaje de “Alerta: tipo-de-alerta”. En este caso el argumento ({0}) que figura es el primer campo (tipo de alerta) del objeto JSON que contiene la información del incidente.
        \item \textit{alert\_subject\_args}: contiene el nombre del campo que será utilizado por \textit{alert\_subject}
        \item \textit{alert\_text}: contiene el mensaje de la notificación, se decidió presentar en una lista los parámetros más importantes de los incidentes. Estos incluyen el nombre de la alerta (que también está en el asunto del correo), timestamp indicando la fecha y hora de la detección del incidente, dirección IP de destino, el puerto de destino, dirección IP y puerto de origen, la interfaz del sensor que realizó la detección, información sobre la firma del incidente y un link para verlo en kibana.
        \item \textit{alert\_text\_type}: especifica el formato del texto de la alerta, en este caso optamos por utilizar la sintaxis de formato standard de Python.
        \item \textit{alert\_text\_args}: contiene los nombres de los campos cuyos valores serán utilizados para el contenido del mensaje.
    \end{itemize}
    \begin{figure}[H]
    \centering
        \includegraphics[width=1\textwidth]{./iteracion_2_imagenes/5-codigoAlerta3.png}
        \caption{Tercera parte del código de configuración de una regla}
        \label{fig:iter2_3_codigo}
    \end{figure}
    \FloatBarrier
    En la Figura \ref{fig:iter2_4_codigo} se observan los campos correspondientes al envío de la notificación:
    \begin{itemize}
        \item \textit{alert}: determina el tipo de alerta implementada, es decir, el método de envío. ElastAlert puede enviar mensajes y notificaciones por varios medios que incluyen el correo electrónico, servicios y aplicaciones, como Telegram, JIRA, entre otros. En este proyecto se decidió utilizar el correo electrónico de la organización como medio de notificación de alertas.
        \item \textit{email}: campo que contiene el correo electrónico de destino. Puede ser una dirección o una lista de direcciones de correos.
        \item \textit{from\_addr}: este campo contiene el correo electrónico remitente que ElastAlert utilizará para enviar las notificaciones.
        \item \textit{smtp\_host}: servidor SMTP del correo utilizado para enviar las notificaciones.
        \item \textit{smtp\_port}: puerto utilizado por el servidor SMTP.
        \item \textit{generate\_kibana\_link}: variable booleana que de estar activada genera un tablero temporal de Kibana y un link para acceder a el.
        \item \textit{use\_kibana4\_dashboard}: link hacia un tablero de kibana (versión 4).
    \end{itemize}
    \begin{figure}[H]
    \centering
        \includegraphics[width=1\textwidth]{./iteracion_2_imagenes/6-codigoAlerta4.png}
        \caption{Cuarta parte del código de configuración de una regla}
        \label{fig:iter2_4_codigo}
    \end{figure}
    \FloatBarrier
    En la Figura \ref{fig:iter2_diagrama_envio_alertas} se observa un diagrama de secuencia del envío de una alerta.
    La secuencia comienza cuando los logs que llegan a Logstash son filtrados y enviados de forma estructurada (JSON) al puerto 9200 de Elastisearch. ElastAlert, por otro lado, realiza consultas por logs a Elasticsearch. Cuando ElastAlert recibe una respuesta a su petición, procede a analizar los resultados en búsqueda de patrones que se puedan identificar en los logs recibidos. \par
    Los patrones a buscar están definidos en las reglas habilitadas de ElastAlert y si se produce una coincidencia, se procede a enviar una notificación por los medios especificados en las reglas. Como se mencionó anteriormente, en este proyecto el medio elegido fue el correo electrónico de la organización. Independientemente de la coincidencia o no de algún patrón y la eventual notificación de alerta a los correspondientes responsables, ElastAlert vuelve a consultar a Elasticsearch por otros logs y procede a repetir el procedimiento descrito, mientras este activo como servicio.\par 
    \begin{figure}[H]
    \centering
        \includegraphics[width=1\textwidth]{./iteracion_2_imagenes/2-diagrama-de-secuencia-envio-alerta.png}
        \caption{Diagrama de secuencia del envío de una alerta}
        \label{fig:iter2_diagrama_envio_alertas}
    \end{figure}
    \end{section}

\label{iteracion2}

    \chapter{Iteración III: “Priorización de alertas”}
    En Security Onion y otros sistemas, el elemento descriptor que identifica y procesa a cada definición de incidente en particular es la regla. Las reglas comprenden una serie de campos que describen con precisión la naturaleza de un incidente dado y por lo tanto, existen tantas reglas como amenazas en circulación. \par
    Cuando un nuevo malware es descubierto por el equipo de algún CSIRT con la capacidad de investigación suficiente o reportado a un laboratorio apropiado para este fin, es posible realizar un estudio de sus características y una vez identificadas estas últimas, proceder a crear una regla y agregarla al repositorio correspondiente para que otros CSIRT actualicen sus IDS con esta nueva definición y así contar con un filtro (la regla) que permita detectar este malware. Las reglas tienen un conjunto de campos donde se detallan características del paquete y su contexto, tales como el puerto de origen y destino, protocolo empleado, dirección IP, etc y unos campos dedicados a la naturaleza del incidente (clasificación, mensaje, prioridad, etc). Algunos de estos campos son comunes a todas las reglas y permiten agruparlas para administrar eficientemente las alertas generadas cuando una regla coincide con la descripción de un incidente. Dado que estos campos también se pueden considerar observables, es posible utilizarlos por TheHive y Cortex para automatizar respuestas. \par

    \begin{section}{Análisis de prioridades de los incidentes}
    Como se indicó anteriormente, la estructura de las reglas consisten en dos partes bien definidas: un encabezado (header) que es obligatorio  y un conjunto de campos opcionales. Dentro del header encontramos la acción (alerta, notificación, etc), el protocolo (tcp, udp), puertos de origen y destino, el sentido del evento (entrante o bidireccional) y las direcciones IP de origen y destino. \par
    La segunda parte de las reglas incluye dos tipos de campos: los que describen la naturaleza del evento y aquellos que contienen información del paquete de datos. Dentro del primer grupo encontramos aquellos tales como msg (descripción del evento), sid (id de la firma), classtype (clasificación de reglas o alertas), priority (prioridad de la firma y/o alerta), target (especifica de qué lado está el objetivo, es decir puerto de origen y puerto de destino), entre otros. El segundo grupo contiene datos extraídos que provienen desde de la capa de red hasta la de aplicación de la pila OSI. Se pueden mencionar a los campos “GeoIP” (localización geográfica de la IP), “Fragbits” (presencia del bit de fragmentación), “ACK” (presencia del campo ACK en paquete TCP), “itype” (número del tipo de mensaje ICMP), “http.method” (tipo de método HTTP usado), entre otros.
     \begin{figure}[H]
        \centering
        \includegraphics[width=0.7\textwidth]{./iteracion_3_imagenes/figura_41_estructura_regla.png}
        \caption{Estructura general de una regla}
        \label{fig:figura_41_estruc_regla}
     \end{figure}
    
    Como se describió en los párrafos precedentes, como los campos están presentes en todas las reglas, es posible hacer uso de algunos de ellos para agrupar reglas que describen amenazas pertenecientes a un mismo grupo o categoría de malware, intentos de intrusión, reconocimiento, escalado de privilegios, etc y por lo tanto son útiles para gestionar los incidentes. \par
	Es posible configurar esta gestión a través de un archivo que relaciona campos como categorías de eventos con prioridades de la alerta generada. Este archivo llamado “classification.config” se encuentra bajo el directorio que almacena las reglas descargadas desde diversas fuentes; en particular relaciona los campos “classtype” con “priority”, de manera tal que cualquier regla cuyo campo classtype contenga a los descritos en este archivo, generará una alerta con prioridad definida también en este. De esta manera, es posible administrar un enorme número de reglas agrupadas en un reducido grupo de categorías y modificar el nivel de prioridad que tendrá en el sistema las alertas que generan. \par
	El objetivo de asignar distintos niveles de prioridad a las alertas generadas por los eventos que sucedan radica en la naturaleza de los eventos, su importancia y la gestión de la atención de los analistas del CSIRT. Esto se debe a las necesidades de optimizar el uso de los recursos técnicos y humanos del centro de respuesta a incidentes para cumplir de la manera más eficiente posible con los objetivos y políticas de la organización a la cual pertenece. De esta manera, la naturaleza de los incidentes determina su elegibilidad para una respuesta automatizada al tener en cuenta por un lado su estructura bien conocida y por el otro su alta tasa de repetición en un periodo determinado. En estos casos, sería inutil destinar valiosos recursos como la atención de un analista ya que conoce perfectamente la estructura del incidente y por lo tanto la respuesta apropiada o en aquellos casos en los que aún conocida su estructura, el incidente proviene en simultáneo de múltiples fuentes en muy poco tiempo, de manera que la capacidad humana de responder de a uno a la vez estaría tan sobrepasada que no sería efectiva. Estos son los casos de ataques de reconocimiento y los de denegación distribuida de servicio, entre otros. \par
	De aproximadamente cuarenta y siete (47) categorías de incidentes disponibles por defecto, consideramos para el máximo nivel de prioridad a siete clasificaciones dado su nivel de ocurrencia y nivel de impacto para la organización. 
    \begin{itemize}
        \item Web-application-attack: esta categoría engloba a un conjunto enorme de malware y ataques a nivel de capa de aplicación. Gusanos, ransomware, ataques de reconocimiento entre otras amenazas comparten esta categoría. Sobre el caso particular de los ataques de reconocimiento, se aplicaron filtros para separarlos de los demás ya mencionados. 
        \item Unsuccessful User: intentos repetidos de ganar acceso en ciertos activos e infraestructura de la organización.
        \item Attempted-dos: intentos de ataque de denegación de servicio y su variante distribuida
        \item Known client side exploit attempt: intento de ejecución de exploits en el lado del cliente.
        \item Exploit Kit Activity Detected: detección de actividad de un kit de exploits
        \item A suspicious filename was detected: detección de nombres de archivos sospechosos
        \item Network Trojan: detección de un virus troyano de red.
    \end{itemize}
    \end{section}
    \chapter{Conclusión}
Haber desplegado una solución SIEM y en consecuencia, dotar al incipiente CSIRT de la Universidad Nacional de Córdoba con herramientas de monitoreo y gestión de eventos de seguridad de la información, permitió sumar nuevas capacidades de ciberseguridad a la Universidad. Estas consisten en poder monitorear el tráfico desde y hacia la universidad, incluyendo sus dependencias, infraestructura y activos de información.
 \par
Se logró desarrollar el sistema de monitoreo mediante el despliegue y configuración de Security Onion. A partir de este momento la Universidad pudo observar y analizar los eventos de seguridad de la información que estaban ocurriendo en su tráfico de datos.  \par
Fueron enfrentados varios desafíos, algunos de los cuales pudieron superarse con éxito y otros permanecen como tareas pendientes para posteriores trabajos. En cuanto a los problemas que pudieron ser resueltos se destacan la priorización de categorías de eventos, el aprendizaje y uso de las correlaciones que resuelve ElastAlert. Además fue posible aprender sobre la creación de filtros de eventos para disparar notificaciones específicas, identificar el comportamiento de determinados tipos de eventos, la administración inteligente del espacio en el disco y el desarrollo de scripts en TheHive. \par
En cuanto a las oportunidades de mejora identificadas para posteriores trabajos, se encuentran: el despliegue y optimización de nodos de almacenamiento, especificar acciones automáticas según el tipo de ataque así como la identificación de las causas del comportamiento anómalo de algunos componentes. También podemos incluir la comprobación automática del estado de los servicios en un tablero y probar cuál es la máxima cantidad de nodos Forward que es posible atender con un nodo Master. \par
Otro hecho a destacar es haber podido implementar este proyecto en un entorno complejo y con hardware de alto rendimiento, como es la infraestructura de redes de datos de la Universidad Nacional de Córdoba. Esto nos permitió experimentar y poner a prueba nuestro proyecto bajo condiciones reales y en un ambiente altamente demandante. Fue posible contemplar el desempeño de la solución y poner a punto el sistema. \par
Finalmente pudimos adquirir conocimientos y experiencia, en el área de la seguridad de redes de la información en particular y en ciberseguridad en general. Ensayamos el proceso, vimos sus efectos y consecuencias, de desplegar este tipo de soluciones a gran escala. Fue una gran experiencia haber explorado esta área profesional y haber adquirido aptitudes para desarrollar en el futuro. \par



    \chapter{\Large Trabajos Futuros }
En cuanto a las oportunidades de mejora identificadas para posteriores trabajos, se encuentran:

\begin{enumerate}
    \item \textbf{Despliegue y optimización de nodos de almacenamiento}: esto se debe a que el volumen de datos generados por el trafico entre las dependencias monitoreadas y el \textit{switch} central es demasiado alto para depender del almacenamiento en un solo nodo (\textit{Master}), por lo que resulta necesario desplegar nodos de almacenamiento (\textit{Storage}) distribuido e implementar servicios de \textit{buffer} y gestión de carga tales como Redis.
    
    \item \textbf{Automatización de acciones según el tipo de ataque}: TheHive y Cortex permiten desencadenar respuestas automáticas, sin embargo el inicio del proceso depende, por defecto, de una orden del analista. TheHive soporta la implementación de \textit{webhooks}, pero durante el transcurso de la experiencia de este Proyecto Integrador, dicha funcionalidad estaba en estado \textit{beta} por parte de sus desarrolladores, por lo que las implementaciones desarrolladas mediante \textit{webhooks} para ejecutar \textit{responders} de Cortex no pasaron de la fase de experimentación.
    
    \item \textbf{Comprobación automática del estado de los servicios en un tablero}: como \textit{Security Onion} cuenta con una suite de aplicaciones seria deseable tener un tablero para poder ver la salud de los servicios y tener la información actualizada minuto a minuto.
    
    \item \textbf{Determinar la máxima cantidad de nodos \textit{Forward} que es posible atender con un nodo \textit{Master}}: Quedo por verificar la cantidad máxima de nodos \textit{Forward} que puede soportar un nodo \textit{Master}. Solo se verifico la cantidad de paquetes que pierde un nodo \textit{Forward} según el volumen de trafico de la red. 

\end{enumerate}
 
    
    % Bibliografía 
    \printbibliography
    %%\chapter*{\Large Anexos}
%\addcontentsline{toc}{chapter}{Anexos}
\appendix
\chapter{\Large Entorno de Virtualización}
    \begin{section}{Configuración del entorno de virtualización}
        Para el entorno de virtualización se utilizó VMWare ESXi \cite{vmware}, concretamente la suite vSphere HyperVisor v6.7.0 u3. Este sistema operativo basado en Unix permite gestionar los recursos de hardware disponibles, almacenar imágenes de distintos sistemas operativos y crear máquinas virtuales con estos últimos. Durante el proceso de creación de una máquina virtual, se selecciona el sistema operativo deseado y es posible asignar distintas cantidades de memoria principal, secundaria, cantidad de vCPU, número y tipo de enlaces de red, entre otros parámetros. \par
         \begin{figure}[H]
          \centering
           \includegraphics[width=0.7\textwidth]{./iteracion_1_imagenes/figura_34_diagrama_VM.png}
            \caption{ Diagrama de una máquina virtual desde el punto de vista de un HyperVisor\cite{vmware}}
            \label{fig:maquina_virtual}
        \end{figure}
        \end{section}
        
\chapter{\Large Configuración del sistema base}
\begin{section}{Introducción}
        Es posible instalar Security Onion en su versión 16.04 de dos maneras: mediante una ISO provista por los desarrolladores o bien mediante una serie de paquetes en una distribución Ubuntu \cite{ubuntu}. En este último caso será necesario contar con la distribución Ubuntu en su versión 16.04, ya que las distribuciones de Security Onion siguen a las distribuciones respectivas de Ubuntu. A partir del año 2020 se lanzaron nuevas versiones de Security Onion con soporte a otras distribuciones Linux: CentOS 7 y Ubuntu 18.04 y 20.04, aunque en el futuro se podrá desplegar en otros tipos de sistema Linux. Desde la versión 2.x de Security Onion en adelante, el sistema se despliega en contenedores.
        \end{section}
        \begin{section}{Instalación y configuración de Security Onion}
        Como se mencionó en la sección anterior, existen dos maneras de instalar esta versión de Security Onion: a partir de una imagen ISO o mediante paquetes. \par 
        En primer lugar se crearon dos máquinas virtuales sobre el hipervisor VMWare ESX(i), con los requisitos definidos en la sección \ref{seleccion_hw} para un nodo Master y Forward.\par
        En las maquinas virtuales se utilizo la ISO de Security Onion. La instalación se realizo mediante el asistente integrado que dispone la imagen del sistema operativo. Este asistente comenzo por configurar la interfaz de red como enlace de administración, como se ve en la Figura \ref{fig:figura_37_sonion_conf}. Se solicito ingresar parámetros para la configuración de IP estática o dinámica, servidor DNS, etc. \par
        \begin{figure}[H]
            \centering
            \includegraphics[width=0.5\textwidth]{./iteracion_1_imagenes/figura_37_sonion_conf.png}
            \caption{Elección de la interfaz de administración}
            \label{fig:figura_37_sonion_conf}
        \end{figure}
        \FloatBarrier
        En segundo lugar, el asistente preguntó si se deseaba configurar una interfaz para monitoreo. Este paso fue importante ya que si lo que se estaba desplegando era un nodo Master en una topología distribuida, habia que elegir la opción de no implementar una interfaz de monitoreo. Cuando tuvimos que desplegar un nodo Forward en una topología distribuida fue necesario seleccionar la opción para configurar una interfaz para monitorear tráfico de red. \par
        Luego de aceptar las modificaciones propuestas por el asistente, el sistema se reinició. Completado el reinicio, fue necesario volver a ejecutar el asistente de instalación, que nuevamente preguntó si se deseaba configurar las interfaces de red. Fue necesario seleccionar la opción de saltar este paso, tras lo cual el asistente preguntó el modo de despliegue del sistema: evaluación o producción, como se ve en la figura \ref{fig:figura_38_sonion_modo}
        \begin{figure}[H]
            \centering
            \includegraphics[width=0.7\textwidth]{./iteracion_1_imagenes/figura_38_sonion_modo.png}
            \caption{El asistente de instalación permitía elegir el modo de despliegue}
            \label{fig:figura_38_sonion_modo}
        \end{figure}
        \FloatBarrier
        %Seleccionado el modo de despliegue, se debió elegir entre un nuevo despliegue de Security Onion o agregar la instalación actual a un entorno existente, como se ve en la figura \ref{fig:figura_38_b_sonion_modo}. Elegida la primer opción, se creó automáticamente un nodo Master, mientras que la opción restante configuraba un nodo Forward si hubiese sido seleccionada. \par
        Seleccionado el modo de despliegue, se debió elegir entre la creación de un nodo Máster o añadir un nuevo nodo Forward a una topología preexistente, como se ve en la figura \ref{fig:figura_38_b_sonion_modo}. Elegida la primer opción, se creó automáticamente un nodo Master, mientras que la opción restante configuraba un nodo Forward si hubiese sido seleccionada. \par
        \begin{figure}[H]
            \centering
            \includegraphics[width=0.7\textwidth]{./iteracion_1_imagenes/figura_38_b_sonion_modo.png}
            \caption{Elección entre la creación de un nodo Máster o añadir un nuevo nodo Forward a una topología preexistente}
            \label{fig:figura_38_b_sonion_modo}
        \end{figure}
        El asistente solicito el ingreso de datos para crear una cuenta de usuario y la elección de una configuración automática o manual para el resto de las opciones: persistencia de la base de datos mysql (por defecto 30 días), número de días de respaldo en caso de falla (por defecto 7 dias), selección de conjunto de reglas de detección (Emerging Threats Open o versiones de pago), motor de IDS (suricata o snort), habilitación de los sensores IDS (no recomendado para el nodo Master), SALT y activar o no la pila Elastic. Por último se debió seleccionar el almacenamiento de los logs: localmente o en un nodo de almacenamiento, así como el tamaño límite.\par
        Posteriormente se desplegó un nodo Forward. Las configuraciones ofrecidas por el asistente fueron muy similares a las descritas para el nodo Master. Los cambios que se requirieron fueron los datos relacionados a la configuración de red del enlace al nodo Master, por lo que fue necesario que previamente exista este nodo. Finalmente, fue necesario habilitar la conexión ssh e ingresar las credenciales del nodo Master, para la comunicación mediante un autotunel ssh entre ambos nodos.\par
        Con el fin de realizar pruebas iniciales del sistema, se eligió el despliegue mediante una ISO. Posteriormente, cuando se comprobó el funcionamiento de la mayoría de sus componentes, se desplegó el sistema mediante paquetes de la distribución 16.04 de Security Onion. Esta decisión se tomo debido a una solicitud de la organización, ya que esta contaba con maquinas virtuales configuradas con Ubuntu 16.04. \par
        Se dispuso de un sistema operativo Ubuntu Server 16.04 con la particularidad de tener dos discos montados: el principal para el sistema operativo y el secundario para los datos recolectados en un directorio \textbf{/nsm}. En el caso de un servidor Master se almacenaron índices en este directorio. En el caso de un nodo Forward, en el directorio \textbf{/nsm} se resguardaron capturas de paquetes o logs. \par 
        Sin importar el método de instalación, al finalizar el proceso, las configuraciones que se mencionaron anteriormente también se pudieron llevar a cabo mediante el comando \textit{sosetup -f $\sim$/sosetup.conf}. Esta instrucción ejecuta un programa que lee las configuraciones previamente guardadas en un archivo.\par
        Con el objetivo de cumplir uno de los requerimientos no funcionales del proyecto, que implica la automatización del despliegue (instalación y configuración) del sistema, se utilizó una herramienta de administración automatizada de servidores llamada Ansible en su versión 2.8.4 para la cual se desarrollaron scripts YAML conteniendo la secuencia de instalación de los paquetes, configuraciones, rol del nodo (Forward o Master) y librerías requeridas para el apropiado funcionamiento del sistema.\par
        \end{section}
        
        \begin{section}{Instalación y configuración de TheHive - Cortex}
        Para la instalación del gestor de incidentes, que tiene como componentes a  TheHive y Cortex, se utilizó  el sistema operativo Debian 10. En primer lugar se instaló TheHive, para ello fue necesario realizar la instalación previa de los componentes necesarios como las librerías de Java 8, Python (versiones 2.7 y 3.6) y Elasticsearch; este último requirió una configuración en su archivo elasticsearch.yaml, como se ve en la Figura \ref{fig:figura_39_thehive_conf}:
        
        \begin{figure}[H]
            \centering
            \includegraphics[width=0.7\textwidth]{./iteracion_1_imagenes/figura_39_thehive_conf_elastic.png}
            \caption{Configuración añadida a elasticsearch.yaml para la instalación de TheHive}
            \label{fig:figura_39_thehive_conf}
        \end{figure}
        Las modificaciones realizadas a este archivo fueron:
        \begin{itemize}
            \item network.host: dirección IP del servidor que contiene Elasticsearch, en este caso es el mismo por lo que se indico la dirección local.
            \item cluster.name: nombre del cluster en el que se desplegó Elasticsearch.
            \item Thread\_pool.index.queue\_size: tamaño de la cola de índices.
            \item Thread\_pool.search.queue\_size: tamaño de la cola para operaciones de búsqueda pendientes.
            \item Thread\_pool.bulk.queue\_size: tamaño de la cola de operaciones de escritura pendientes para un documento.
        \end{itemize}
        \par
        Finalmente, los últimos pasos para la instalación de TheHive consistieron en habilitar e iniciar el servicio de elasticsearch, agregar el repositorio que contiene los paquetes de TheHive, instalarlo y luego habilitar el servicio para poder iniciarlo. \par
        En cuanto a Cortex, el proceso fue similar al anteriormente descrito para TheHive, donde una vez descargados e instalados los paquetes de Cortex con su correspondiente secuencia de habilitación e inicio; se procedió a descargar del repositorio los responders y analyzers respectivos. Por último, se modificó el archivo de configuración de Cortex para indicar la ubicación del directorio que contiene los responders y analyzers mencionados anteriormente. \par
        Posteriormente se actualizó la base de datos elasticsearch mediante la GUI web de Cortex, se creó un superusuario y luego las organizaciones donde se administraron usuarios comunes y analyzers; fue necesario crear un usuario con el rol de administrador de organizaciones. Las organizaciones tuvieron habilitados y configurados determinados responders y analyzers según fue necesario. \par
        El último paso del proceso consistio en comunicar TheHive y Cortex entre sí. Para ello se generó una API key en Cortex que fue usada como parte de las modificaciones necesarias al archivo application.conf de TheHive. Las modificaciones completas que se realizaron al mencionado archivo se pueden apreciar en la Figura \ref{fig:figura_40_thehive_cortex_conf}:\par
        
        \begin{figure}[H]
            \centering
            \includegraphics[width=0.7\textwidth]{./iteracion_1_imagenes/figura_40_thehive_cortex_conf.png}
            \caption{Modificación al archivo application.conf de TheHive para la comunicación con Cortex}
            \label{fig:figura_40_thehive_cortex_conf}
        \end{figure}
        Los campos modificados que se pueden observar son:
        \begin{itemize}
            \item play.modules.enabled: habilita el módulo de Cortex
            \item url: especifica la dirección url del servidor de Cortex
            \item key: Es API key del usuario de Cortex.
            \item refreshDelay: periodo de tiempo de comprobación de casos generados por TheHive
            \item maxRetryOnError: número máximo de intentos de recuperación de un error
            \item statusCheckInterval: periodo de tiempo de comprobación del estado de Cortex
        \end{itemize}
        \end{section}
    
\chapter{\Larger Configuracion de ElastAlert}
Se escribieron archivos de configuración para cada tipo de evento, de manera tal que las reglas que disparan cada alerta lo hagan según el comportamiento y naturaleza del incidente. En las figuras \ref{fig:iter2_1_codigo} a \ref{fig:iter2_4_codigo} se observa el código desarrollado para el envío de alertas en caso de producirse un reconocimiento y escaneo de puertos. Las figuras corresponden al mismo archivo de configuración, pero cada una muestra una parte del código para facilitar la explicación del mismo.\par
    En la Figura \ref{fig:iter2_1_codigo} se muestra la primer parte del código de configuración de una regla:
    \begin{itemize}
        \item \textit{es\_host}: el nombre del host de la base de datos elasticsearch.
        \item \textit{es\_port}: el número de puerto del host de elasticsearch mediante el cual ElastAlert se contactara con la base de datos.
        \item \textit{name}: nombre de la regla. Debe ser único.
        \item \textit{type}: especifica el tipo de regla. En este caso es del tipo “frecuency”, que determina que esta regla se activará si al menos se produce cierto número de eventos en determinado intervalo de tiempo.
        \item \textit{num\_events}: especifica la cantidad de eventos necesarios para activar este tipo de regla.
        \item \textit{timeframe}: determina el marco temporal necesario para este tipo de reglas. Tiene un sub parámetro que especifica la unidad temporal y la cantidad de estas. En este caso optamos por un marco temporal de un (1) minuto.
    \end{itemize}
    \begin{figure}[H]
    \centering
        \includegraphics[width=0.7\textwidth]{./iteracion_2_imagenes/3-codigoAlerta-1.png}
        \caption{Primera parte del código de configuración de una regla}
        \label{fig:iter2_1_codigo}
    \end{figure}
    \FloatBarrier
    En la Figura \ref{fig:iter2_2_codigo} se muestran los campos implicados en la consulta a elasticsearch:
    \begin{itemize}
        \item \textit{index}: es el índice a consultar
        \item \textit{use\_strftime\_index}: es una variable booleana que, en el caso de ser verdadera, da formato al índice utilizando “\textit{datetime\_strftime}” para cada consulta. Este último método de Python crea un string representando el tiempo en un formato específico. Se utiliza en el caso de que la consulta realizada incluya varios días, con lo que los índices se concatenan utilizando “;” permitiendo una búsqueda más eficiente.
        \item \textit{filter}: determina los filtros de elasticsearch utilizados para la consulta. Utiliza los subparametros “term”  y “category” para indicar la clave a buscar. En este caso utilizamos “category: scan” para filtrar todos los eventos que pertenezcan a la categoría del reconocimiento de puertos.
        \item \textit{query\_term}: permite agrupar las consultas, en este caso agrupamos las alertas resultantes por direcciones IP de origen y destino, así como todos los eventos que tengan el campo “alert”.
        \item \textit{realert}: ignora las alertas repetidas en un cierto periodo de tiempo. En la práctica esto permitió evitar saturar con notificaciones a los responsables de los activos afectados, ya que el reconocimiento de puertos es un incidente extremadamente común y periódico, produciéndose muchas veces por día.
    \end{itemize}
    \begin{figure}[H]
    \centering
        \includegraphics[width=1\textwidth]{./iteracion_2_imagenes/4-codigoAlerta2.png}
        \caption{Segunda parte del código de configuración de una regla}
        \label{fig:iter2_2_codigo}
    \end{figure}
    \FloatBarrier
    En la Figura \ref{fig:iter2_3_codigo} se observan los campos utilizados para armar el cuerpo del mensaje y el asunto:
    \begin{itemize}
        \item \textit{doc\_type}: especifica el tipo de documento a consultar en elasticsearch.
        \item \textit{alert\_subject}: permite personalizar el mensaje de una alerta agregando un pequeño resumen. En este proyecto lo utilizamos en todas las reglas, para que el “asunto” del correo electrónico contenga un mensaje de “Alerta: tipo-de-alerta”. En este caso el argumento ({0}) que figura es el primer campo (tipo de alerta) del objeto JSON que contiene la información del incidente.
        \item \textit{alert\_subject\_args}: contiene el nombre del campo que será utilizado por \textit{alert\_subject}
        \item \textit{alert\_text}: contiene el mensaje de la notificación, se decidió presentar en una lista los parámetros más importantes de los incidentes. Estos incluyen el nombre de la alerta (que también está en el asunto del correo), timestamp indicando la fecha y hora de la detección del incidente, dirección IP de destino, el puerto de destino, dirección IP y puerto de origen, la interfaz del sensor que realizó la detección, información sobre la firma del incidente y un link para verlo en kibana.
        \item \textit{alert\_text\_type}: especifica el formato del texto de la alerta, en este caso optamos por utilizar la sintaxis de formato standard de Python.
        \item \textit{alert\_text\_args}: contiene los nombres de los campos cuyos valores serán utilizados para el contenido del mensaje.
    \end{itemize}
    \begin{figure}[H]
    \centering
        \includegraphics[width=1\textwidth]{./iteracion_2_imagenes/5-codigoAlerta3.png}
        \caption{Tercera parte del código de configuración de una regla}
        \label{fig:iter2_3_codigo}
    \end{figure}
    \FloatBarrier
    En la Figura \ref{fig:iter2_4_codigo} se observan los campos correspondientes al envío de la notificación:
    \begin{itemize}
        \item \textit{alert}: determina el tipo de alerta implementada, es decir, el método de envío. ElastAlert puede enviar mensajes y notificaciones por varios medios que incluyen el correo electrónico, servicios y aplicaciones, como Telegram, JIRA, entre otros. En este proyecto se decidió utilizar el correo electrónico de la organización como medio de notificación de alertas.
        \item \textit{email}: campo que contiene el correo electrónico de destino. Puede ser una dirección o una lista de direcciones de correos.
        \item \textit{from\_addr}: este campo contiene el correo electrónico remitente que ElastAlert utilizará para enviar las notificaciones.
        \item \textit{smtp\_host}: servidor SMTP del correo utilizado para enviar las notificaciones.
        \item \textit{smtp\_port}: puerto utilizado por el servidor SMTP.
        \item \textit{generate\_kibana\_link}: variable booleana que de estar activada genera un tablero temporal de Kibana y un link para acceder a el.
        \item \textit{use\_kibana4\_dashboard}: link hacia un tablero de Kibana (versión 4).
    \end{itemize}
    \begin{figure}[H]
    \centering
        \includegraphics[width=1\textwidth]{./iteracion_2_imagenes/6-codigoAlerta4.png}
        \caption{Cuarta parte del código de configuración de una regla}
        \label{fig:iter2_4_codigo}
    \end{figure}
    \FloatBarrier
    
\chapter{Modificación de un filtro de Logstash}
El proceso de creación de este filtro consistió en copiar el archivo de configuración de Filebeat: \textit{9500\_output\_beats\_custom.conf} que se encuentra en el directorio mencionado anteriormente. El archivo fue copiado a la carpeta \textit{/etc/logstash/custom}, donde fue modificado para crear el filtro con Grok. Este fue creado teniendo en cuenta el cuerpo de los mensajes provenientes del archivo de logs analizado, dado que se conocían los campos de interés que contenían los logs y que por lo tanto se deseaba extraer. Una vez finalizada la modificación del archivo, se reinicio el servicio de Logstash y se comprobó que el archivo \textit{logstash.log} para verificar que no hubiera errores.\par


    


 
  


    

\end{document}
