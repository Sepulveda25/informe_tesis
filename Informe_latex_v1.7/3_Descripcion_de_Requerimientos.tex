\chapter{Descripción de Requerimientos}
    Con el objetivo de desplegar un sistema SIEM capaz de soportar los requerimientos funcionales y no funcionales, es necesario definir el entorno en el que operará la plataforma. \par
    Para esto se requiere, en primer lugar describir la topología de la red de la organización: realizar un relevamiento de las conexiones  de la infraestructura de red interna de la unidad central, las de sus dependencias y la red entre las unidades geográficamente distribuidas si las hubiera. Debe incluirse la topología de las conexiones de salida a Internet. \par
    %En segundo lugar será necesario inventariar los activos de la organización: se requerirá una investigación y un relevamiento de los activos con los que cuentan las infraestructuras de red y de datos a fin de clasificarlos. \par
    Las tareas de relevamiento anteriormente descritas proporcionarán un entendimiento acabado y profundo de la situación en la que se encuentra la infraestructura. Como resultado, será posible identificar puntos críticos a tener en consideración y como consecuencia, elegir la solución que mejor se ajuste a las necesidades de la organización. \par

    \begin{section}{Requerimientos funcionales del SIEM}
    \begin{enumerate}
        \item Recolectar y almacenar datos de incidentes de seguridad en la infraestructura de la red corporativa.
        \item Recolectar y almacenar información contextual y asociada a los activos vinculados  al incidente.
        \item Visualizar las alertas en un tablero de mando. 
        \item Implementar un sistema de envío de alertas de seguridad que notifique a los responsables de activos de información afectados.
        \item Definir un criterio para priorizar alertas.
        \item Implementar un sistema de correlación de alertas de seguridad.
    \end{enumerate}
        
    \end{section}
    
    \begin{section}{Requerimientos no funcionales}
    \begin{enumerate}
        \item La solución propuesta debe utilizar software libre.
        \item El sistema operativo base debe ser tipo Unix y abierto.
        \item La arquitectura de la solución debe ser escalable a demanda de la organización.
        \item Se requiere un despliegue automatizado de la solución.
    \end{enumerate}

    \end{section}
    
    \begin{section}{Análisis de riesgo}
    En primer lugar, será seleccionada una plataforma cuyo código sea libre y abierto. Posteriormente será elegido un sistema operativo libre, tipo Unix, que sea compatible con la solución escogida. Se tendrá en cuenta el desarrollo de una arquitectura de despliegue que contemple la escalabilidad horizontal de la solución para adaptarse a las necesidades de la organización. \par
    Se adaptará la solución para recolectar y almacenar los datos pertinentes a los incidentes de seguridad que ocurren en la red corporativa, así como la información de contexto de los activos de información que se ven afectados.  Luego de haber recibido y almacenado los datos, se configurará la solución para visualizar la información disponible. Esto permitirá implementar un sistema de envío de alertas que notifiquen a los responsables de los activos de información que se viesen comprometidos. \par
    Finalmente, se dispondrá de un sistema que permita correlacionar y filtrar alertas, en base a políticas a definir que incluyan las categorías de eventos y sus prioridades asociadas.\par
	%Finalmente se procederá a automatizar el proceso completo que comprende la implementación, despliegue y configuración de la solución junto a sus componentes asociados, mediante el uso de herramientas de automatización de tecnologías de la información. \par
    \end{section}
    
    