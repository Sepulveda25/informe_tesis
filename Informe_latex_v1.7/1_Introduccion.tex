\chapter{Introducción}
    \begin{section}{Sistema SIEM}
        Este proyecto consiste en la implementación de un Sistema de Administración de Eventos y Seguridad de la Información (SIEM, por sus siglas en inglés) para la Universidad Nacional de Córdoba. Un SIEM consiste en varias herramientas como bases de datos, filtros para normalizar la información, tablero para visualizar los datos y generador de alertas entre otras. Por otro lado tenemos el monitoreo de la red en tiempo real que utiliza un Sistema de Detección de Intrusiones (IDS, por sus siglas en inglés). Este último envía los datos ya procesados al SIEM para que los almacene en la base de datos. \par
        Además como se pretende que el SIEM funcione dentro de un Equipo de Respuesta a Incidentes de Seguridad Informática (CSIRT, por sus siglas en inglés) se necesita integrar un gestor de incidentes. Este último sirve para tener un registro de los incidentes ocurridos, permite administrar las tareas del equipo de analistas, compartir y solicitar información con otros CSIRT entre otras funciones.
    \end{section}
    \begin{section}{Objetivo general}
        El objetivo de esta tesis es el desarrollo e implantación de un sistema SIEM dentro del proyecto general de la creación del CSIRT de la Universidad Nacional de Córdoba, con el fin de otorgar al mencionado centro de respuesta, el instrumento capaz de obtener, analizar y presentar datos sobre las amenazas detectadas por los demás subsistemas del CSIRT.
    \end{section}
    
    \begin{section}{Motivación}
        La tecnología y la digitalización de la información convierten a los datos en un activo muy importante de las organizaciones y de los individuos en general. Es fundamental saber cómo proteger los datos para evitar ser víctima de un ciberdelito o parte involuntaria de una ciber operación a gran escala. A pesar de que actualmente las técnicas de seguridad hacia los datos y la infraestructura de redes están en auge, las herramientas de seguridad como firewalls, IDS y otras que permiten prevenir ataques informáticos no son suficientes para mitigar y tener un seguimiento de actividades maliciosas o potencialmente maliciosas para lograr fortalecer la infraestructura y prevenir futuros incidentes. Resulta necesario contar con un sistema global que permita integrar un variado conjunto de utilidades que brindan soluciones puntuales y específicas, para crear una defensa inteligente y eficiente de los activos de información de una organización. \par
        Actualmente la infraestructura de red y los sistemas asociados conviven en un ambiente de saturación de la información que implica un alto costo de procesamiento y ponen a prueba permanentemente a los sistemas encargados de la optimización de los recursos de hardware y software con los que cuenta la infraestructura, tales como uso de CPU y memorias RAM de routers, switches y servidores, el almacenamiento secundario donde el desafío de retener un ingente volumen de datos generados por el exponencial y siempre creciente tráfico de la red amenaza constantemente con el colapso de los medios disponibles sin importar su capacidad de almacenamiento, entre otros problemas, configuran una avalancha constante de información que sería imposible de analizar siquiera una parte de ella en un momento determinado utilizando métodos que impliquen el procesamiento en bruto.  \par
        En esta situación, sería imposible distinguir un evento puntual y nocivo dentro de esta cantidad gigantesca de información que se genera permanentemente en la red, de un evento normal o de tráfico legítimo y en caso de identificar un potencial incidente, este tendría unas probabilidades muy altas de ser un falso positivo. Esto último es característico de los sistemas basados en el análisis de firmas, como los IDS, IPS o antivirus. Es necesario diseñar, desarrollar, implementar, configurar y probar un sistema capaz de orquestar un gran abanico de herramientas diseñadas cada una con un objetivo puntual, combinando las capacidades de todos sus subsistemas para identificar eficientemente las amenazas reales y responder en consecuencia, minimizando los falsos positivos y daños colaterales.

    \end{section}