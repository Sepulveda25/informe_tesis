\documentclass[12pt,twoside]{report}
\usepackage[utf8]{inputenc}
%Para qué los títulos aparezcan en español
\usepackage[spanish]{babel}
\usepackage{csquotes}
\usepackage{array} % para hacer tablas
\usepackage{graphicx} % gráficos
\usepackage{float} % posiciona las imágenes
\usepackage{blindtext}
\usepackage{caption}
\usepackage{placeins}
\usepackage{subfig}
\usepackage[export]{adjustbox}
\raggedbottom

\usepackage[
backend=biber,
sorting=none
]{biblatex} %se puede especificar otra forma de imprimir las citaciones aca
\addbibresource{referencias.bib}

%Margenes según normas APA
\usepackage[a4paper,top=1in, bottom=1in, left=1in, right=1in]{geometry}

% Links a todo el documento y urls
\usepackage{hyperref}
\hypersetup{
    colorlinks=true,
    linkcolor=blue,
    filecolor=magenta,      
    urlcolor=black,
    citecolor=black,
    pdftitle={Sharelatex Example},
    %bookmarks=true,
    pdfpagemode=FullScreen,
}

\begin{document}

    \begin{titlepage}

    \begin{center}

        \begin{tabular}{ p{0.5\textwidth} p{0.5\textwidth} }
        		\centering\includegraphics[height=0.15\textwidth]{./imagenes_caratula/unc_logo.jpg} &
        		\centering\includegraphics[height=0.15\textwidth]{./imagenes_caratula/logo_fcefyn_nuevo.jpg}
        \end{tabular}
        
        \vspace*{30mm}
        \begin{Huge}
            Universidad Nacional de Córdoba \\
        \end{Huge}
        
        \vspace*{10mm}
        \begin{Large}
            Facultad de Ciencias Exactas Físicas y Naturales \\
        \end{Large}
        
        \vspace*{20mm}
        \begin{large}
            Proyecto Integrador: \\
        \end{large}
        
        \vspace*{5mm}
        \begin{LARGE}
            \textbf{Implementación de una solución para gestión de eventos de seguridad de una red de datos} \\
        \end{LARGE}
        
        \vspace*{0.3in}
        \begin{large} \begin{center}
          Autores  
        \end{center}
        \end{large}
        \begin{center}
           \begin{table}[h]
            \resizebox{\textwidth}{!}{%
            \begin{tabular}{|rll|}
            \hline
            \multicolumn{1}{|l}{} & Figueroa Sergio David        & Sepulveda Federico Nicolas       \\
            Correo:               & sergiofigueroa@mi.unc.edu.ar & federico.sepulveda@mi.unc.edu.ar \\
            Matricula:            & 36355236                     & 35037929                         \\
            Telefono:             & 3512941114               & 2994165771                   \\ \hline
            \end{tabular}%
            }
            \end{table}
        \end{center}
        \vspace*{0.3mm}
        \rule{100mm}{0.1mm}\\
        \vspace*{0.3mm}
        \begin{large}
            Director: \\
            Mgter. Ing. Miguel Ángel Solinas \\
        \end{large}
        

    \end{center}

\end{titlepage}

    \chapter*{\Large Resumen}
\addcontentsline{toc}{chapter}{Resumen}

Los servicios digitales de procesamiento y almacenamiento de información se transformaron en la columna vertebral de organizaciones de todo tipo, ya que estas volcaron masivamente su actividad a Internet en la última década. Esto genera un sinnúmero de oportunidades para eventuales atacantes, que buscan detectar y explotar las vulnerabilidades de los sistemas en los cuales se apoya la infraestructura de las organizaciones. \par
El proceso de monitoreo de la seguridad de una red de datos compleja requiere recopilar diferentes tipos de datos para detectar, verificar y contener acciones ofensivas. Es necesario contar con un sistema SIEM que centralice la información proveniente de múltiples fuentes, distribuidas a lo largo de toda la red de datos. En este trabajo se propuso, implementó y desplegó una solución SIEM en la Universidad Nacional de Córdoba.\par

















    % Índice general
    \tableofcontents
    % Índice de figuras
    \listoffigures
    % Índice de tablas
    \listoftables
    % Agradecimientos
    \chapter*{\Large Agradecimientos}
\addcontentsline{toc}{chapter}{Agradecimientos}

    % Glosario
    \chapter*{Glosario}
\addcontentsline{toc}{chapter}{Glosario}

\textbf{CCNA}: acrónimo en inglés de Cisco Certified Network Associate, un certificado de validación profesional emitido por la corporación Cisco para técnicos que operan sus productos. \par

\textbf{CERT}: siglas en inglés de Computer Emergency Response Team, en español equipo de respuesta a incidentes de computadoras. Término registrado comercialmente por la universidad estadounidense de Carnegie Mellon. \par

\textbf{CPU}: siglas en inglés de Central Processing Unit, en español Unidad de Procesamiento Central.\par

\textbf{Creative Commons}: licencia que permite a cualquier usuario copiar, reproducir, adaptar, distribuir, traducir y desarrollar los contenidos multimedia sin costo alguno. La utilización de contenido que se encuentra bajo esta licencia implica reconocer al autor original.\par

\textbf{CSIRT}: siglas en inglés de Computer Security Incident Response Team, en español equipo de respuesta a incidentes de seguridad de computación. Es el equipo de profesionales, sistemas y toda la infraestructura (hardware y software) de detección y respuesta a incidentes de ciberseguridad de una organización.\par

\textbf{DDoS}: siglas en inglés de Denied Distribution Of Service, en español denegación distribuida de servicio. \par

\textbf{DNS}: siglas en inglés de Domain Name Service, en español Servicio de Nombres de Dominio, es un protocolo de red de la capa de aplicación. \par


\textbf{Dirección MAC}: siglas en inglés de Media Access Control, en español control de acceso a medios, es un conjunto de bytes que constituyen la dirección física (única) que identifica a un dispositivo conectado a una red. \par

\textbf{EMCFFAA}: Estado Mayor Conjunto de las Fuerzas Armadas Argentinas. \par

\textbf{Firewall}: Un firewall es un dispositivo basado en un software o hardware o ambos. Bloquea o permite el tráfico de red, basándose en una serie de reglas dinámicas predefinidas y políticas. \par

\textbf{Gbps}: siglas de gigabit por segundo, es una especificación técnica de la medida del ancho de banda y / o velocidad de transmisión, dependiendo del contexto. \par

\textbf{GNU}: acrónimo recursivo en inglés de “GNU is Not Unix”, en español "GNU No es Unix". \par

\textbf{GPL}: siglas en inglés de General Public Licence, en español licencia pública general, es un tipo de licencia GNU. \par

\textbf{GUI}: siglas en inglés de Graphical User Interface, en español Interfaz gráfica de Usuario. \par

\textbf{HIDS}: siglas en inglés de Host Intrusion Detection System, en español sistema de detección de intrusiones en un host o punto final. \par

\textbf{IDS}: siglas en inglés de Intrusion Detection System, en español sistema de detección de intrusiones. Es un componente de software destinado al procesamiento de firmas basadas en la información recolectada del tráfico de red, mediante una sonda colocada en un enlace de la infraestructura de comunicaciones de datos. Los IDS son un componente vital de un CSIRT debido a que realizan la identificación a priori de eventos en el tráfico de datos y su clasificación como un incidente. \par

\textbf{IMAP}:  siglas en inglés de Internet Message Access Protocol. Este protocolo de aplicación, permite a los usuarios acceder a sus e-mails directamente en el servidor y sólo descargar, hacia la máquina local, los mensajes y archivos adjuntos que le resulten de interés. \par

\textbf{Información normalizada}: el objetivo es modificar los mensajes de diferentes fuentes de manera tal que se adapten a un modelo de datos común. \par

\textbf{Infraestructura de IT}: corresponde a la infraestructura de tecnologías de la información (servidores, switches, routers, etc) de una organización. \par

\textbf{IPV4 e IVP6}: siglas en inglés de los protocolos de Internet versiones 4 y 6, respectivamente. \par

\textbf{IPS}: siglas en inglés de Intrusion Protection System, en español sistema de protección de intrusiones. \par

\textbf{Licencia GFDL}: siglas de GNU Free Documentation License. Está orientado a permitir que un manual, un libro de texto o cualquier  otro documento escrito sea libre en el sentido de su difusión, copias, modificaciones y comercialización. \par

\textbf{Licencia AGPL}: siglas de GNU Affero General Public License. Esta licencia asegura los derechos de autor sobre el software y da permisos legales para la copia, distribución y modificaciones del mismo. En caso de modificaciones se debe poner a disposición de la comunidad el código fuente con dichos cambios. \par

\textbf{Licencia APACHE}: licencia de software libre permisiva creada por la Apache Software Foundation. Se diferencia de otros tipos de licencias ya que no exige copyleft en el software donde se aplica. \par

\textbf{Licencia BSD}: siglas en inglés de Berkeley Software Distribution, licencia de software libre desarrollada en dicha universidad homónima de Estados Unidos. \par

\textbf{Linux}: núcleo de código (kernel) abierto de familias de sistemas operativos del mismo nombre, de software libre. \par

\textbf{Log}: equivalente en inglés a “registro” en español. Término utilizado específicamente para registros de datos con un formato definido. \par

\textbf{Malware}: software malicioso diseñado para identificar y / o explotar vulnerabilidades en  los sistemas de  una víctima: sistemas operativos, drivers, cualquier tipo de software, dispositivos, etc. Las consecuencias implican desde el malfuncionamiento del software o dispositivo afectado, robo o pérdida de información, hasta la inutilización total del hardware o sistema infectado. \par

\textbf{MIT}: siglas en inglés de Massachusetts Institute of Technology, universidad de los Estados Unidos cuyo nombre es usado para un tipo de licencia de código libre desarrollada en esa universidad. \par
\textbf{MSSP}: siglas 
en inglés de managed security service provider, en español proveedores de servicios de seguridad gestionados. Empresas que prestan servicios de seguridad informática a organizaciones. \par

\textbf{NIDS}: siglas en inglés de Network Intrusion Detection System, en español sistema de detección de intrusiones a nivel de red. \par

\textbf{NIPS}: siglas en inglés de Network Intrusion Protection System, en español sistema de protección de intrusiones a nivel de red. \par

\textbf{NSM}: siglas en inglés de Network Security Monitoring, en español monitoreo de seguridad de redes. \par

\textbf{RAM}: siglas en inglés de Random Access Memory, en español memoria de acceso aleatorio. \par

\textbf{Ransomware}: Software malicioso, que en un dispositivo puede bloquear la interfaz de usuario o cifrar las información que se encuentra en el disco y posteriormente solicitarle a la víctima un pago para recuperar los datos. \par
\textbf{SaaS}: siglas en inglés de Software as a Service, en español software como servicio, es un modelo de negocio de software a cuyo despliegue y funcionalidades están disponibles a la medida de la demanda del cliente. \par

\textbf{SNMP}: Simple Network Management Protocol (Protocolo simple de administración de red, por sus siglas en ingles). Protocolo que se ubica en el nivel de aplicacion de la pila de red TCP/IP.  \par

\textbf{STDIN}: siglas en ingles de Standard Input, es la entrada estandar de ingreso de datos a un software.\par

\textbf{SYN}: bit usado en el protocolo TCP para indicar la sincronización del número de secuencia al comienzo de una comunicación utilizando el protocolo antes mencionado. \par

\textbf{TCP}: siglas en inglés de Transmission Control Protocol, en español protocolo de control de transmisión. Uno de los protocolos fundamentales en la comunicación de datos. \par

\textbf{UDP}: siglas en inglés de User Datagram Protocol. Es un protocolo que permite la transmisión sin conexión de datagramas en redes basadas en IP. \par

\textbf{VPN}: siglas en inglés de Virtual Private Network, en español red privada virtual. \par

    % A partir de acá comienza la numeración de los capítulos
    \chapter{\Large Introducción}
    \begin{section}{Sistema SIEM}
        Este proyecto consiste en la implementación de un Sistema de Administración de Eventos y Seguridad de la Información (SIEM, por sus siglas en inglés) para la Universidad Nacional de Córdoba. Un SIEM consiste en varias herramientas como bases de datos, filtros para normalizar la información, tablero para visualizar los datos y generador de alertas entre otras. Por otro lado, tenemos el monitoreo de la red en tiempo real que utiliza un Sistema de Detección de Intrusiones (IDS, por sus siglas en inglés). Este último, envía los datos ya procesados al SIEM para que los almacene en la base de datos. \par
        Además, como se pretende que el SIEM funcione dentro de un Equipo de Respuesta a Incidentes de Seguridad Informática (CSIRT, por sus siglas en inglés) se necesita integrar un gestor de incidentes. Este último, sirve para tener un registro de los incidentes ocurridos, permite administrar las tareas del equipo de analistas, compartir y solicitar información con otros CSIRT entre otras funciones.
    \end{section}
    \begin{section}{Objetivo general}
        El objetivo de este proyecto integrador es el desarrollo e implantación de un sistema SIEM dentro del proyecto general de la creación del CSIRT de la Universidad Nacional de Córdoba, con el fin de otorgar al mencionado centro de respuesta, el instrumento capaz de obtener, analizar y presentar datos sobre las amenazas detectadas por los demás subsistemas del CSIRT.
    \end{section}
    
    \begin{section}{Motivación}
        La tecnología y la digitalización de la información convierten a los datos en un activo muy importante de las organizaciones y de los individuos en general. Es fundamental saber cómo proteger los datos para evitar ser víctima de un ciberdelito o parte involuntaria de una ciber operación a gran escala. A pesar que actualmente las técnicas de seguridad hacia los datos y la infraestructura de redes están en auge, las herramientas de seguridad como \textit{firewalls}, IDS y otras que permiten prevenir ataques informáticos no son suficientes para mitigar y tener un seguimiento de actividades maliciosas o potencialmente maliciosas para lograr fortalecer la infraestructura y prevenir futuros incidentes. Resulta necesario contar con un sistema global que permita integrar un variado conjunto de utilidades que brindan soluciones puntuales y específicas, para crear una defensa inteligente y eficiente de los activos de información de una organización. \par
        Actualmente la infraestructura de red y los sistemas asociados conviven en un ambiente de saturación de la información, que implica un alto costo de procesamiento. Esto ultimo, constituye un desafío constante para los sistemas encargados de la optimización de los recursos de hardware y software con los que cuenta la infraestructura, tales como uso de CPU y memorias RAM de \textit{routers}, \textit{switches} y servidores. Por otro lado, la retención de un ingente volumen de datos generados por el exponencial y siempre creciente tráfico de la red, amenaza constantemente con el colapso de los medios disponibles sin importar su capacidad de almacenamiento. Estos problemas, entre otros, configuran una avalancha constante de información que sería imposible de analizar (siquiera una parte de ella) en un momento determinado utilizando métodos que impliquen el procesamiento en bruto.  \par
        En esta situación, sería imposible distinguir un evento puntual y nocivo dentro de esta cantidad gigantesca de información que se genera permanentemente en la red, de un evento normal o de tráfico legítimo y en caso de identificar un potencial incidente, este tendría unas probabilidades muy altas de ser un falso positivo. Esto último es característico de los sistemas basados en el análisis de firmas, como los IDS, IPS o antivirus. Es necesario diseñar, desarrollar, implementar, configurar y probar un sistema capaz de orquestar un gran abanico de herramientas diseñadas cada una con un objetivo puntual, combinando las capacidades de todos sus subsistemas para identificar eficientemente las amenazas reales y responder en consecuencia, minimizando los falsos positivos y daños colaterales.

    \end{section}
    \chapter{Marco Teórico}
    \begin{section}{Presentación}
        Las infracciones a las políticas de seguridad y los ataques han concentrado la atención sobre las capacidades de detección, investigación y mitigación de incidentes de seguridad de la información en  las organizaciones. Si bien no siempre es posible evitar un incidente de seguridad, es necesario detectar y responder rápidamente para minimizar el daño. Para ello, es preciso realizar inversiones inteligentes basadas en un plan de seguridad que comprenda la realidad y necesidades específicas de la organización, ya que un gran monto de dinero o equipos adquiridos por si mismos no garantizan una mayor protección. \par
        Este plan debe incluir personal especializado, procedimientos e infraestructura  adaptados a la organización, con una gestión de objetivos a cumplir a corto, mediano y largo plazo. \par
        Para las organizaciones que no cuentan con una capacidad de manejo de incidentes, la creación desde cero de un Computer Security Incident Response Team (CSIRT) puede ser un proceso complejo y costoso. Sin embargo, no es necesario una gran inversión para obtener las capacidades elementales ofrecidas por un CSIRT, ya que es posible desarrollar una solución específica y a escala de la organización. \par
        Una vez identificadas las necesidades de la organización, el proceso de creación del CSIRT requiere de la creación, colaboración y comunicación entre los tres pilares que lo componen: el personal, la tecnología y los procesos, como se muestra en la Figura \ref{fig:pilares}. \par
        \begin{figure}[H]
            \centering
            \includegraphics[width=1\textwidth]{./marco_teorico_imagenes/figura_1_pilares.png}
            \caption{Pilares de un CSIRT}
            \label{fig:pilares}
        \end{figure}
        %\par
        El CSIRT debe tener una perspectiva flexible y escalable para mantener el ritmo de las tácticas de los adversarios, acompañando el crecimiento y evolución de la organización. \par
    \end{section}
    
   \begin{section}{Personal}  
   En cuanto al personal, estos comprenden tanto a los encargados de dar respuesta a los incidentes como a los analistas del CSIRT. Si bien la propia organización puede designar a sus integrantes para asumir estas funciones, existen otras alternativas como la tercerización mediante empresas especializadas que proveen el servicio de Managed Security Service Provider (MSSP) o contratar especialistas en respuesta a incidentes en el caso de una emergencia o un problema complejo. Otra vía consiste en la creación de equipos híbridos compuestos por personal perteneciente a la organización y especialistas externos. \par
    De acuerdo a una encuesta del SANS Institute del año 2014 \cite{sans_1}, el 61\% de las organizaciones relevadas manifestaron haber recurrido a personal de emergencia para cubrir incidentes críticos y el 58 \% tenía un equipo de respuesta propio. Por lo que las organizaciones no siempre cubren sus necesidades con miembros de su propio personal y en algunos casos las tareas recaen por completo en los servicios de terceros. Esto se debe a que, sin importar la estructura del equipo, el personal de un CSIRT debe contar con el entrenamiento necesario para tratar con los cambios en las amenazas a las que se enfrenta. En el Cuadro \ref{table:1} se muestran las responsabilidades y la formación requerida para cada uno de los integrantes de un CSIRT. \par
    
    \begin{table}%[ht]
    \centering
        \begin{tabular}{ | m{10em} | m{16em}| m{11em} | } 
            \hline
            Título profesional & Tarea & Entrenamiento requerido \\ 
            \hline
            Nivel 1 - Analista de alertas & Supervisa continuamente la cola de alertas, monitorea el estado de los sensores y los puntos finales, clasifica las alertas de seguridad y recopila los datos necesarios para iniciar el trabajo de Nivel 2. & Procedimientos de triage de alerta y detección de intrusos. Gestión de redes, información de seguridad y eventos. Capacitación en investigación basada en host. \\ 
            \hline
            Nivel 2 - Analista de respuesta a incidentes & Realiza un análisis profundo de incidentes al correlacionar datos de varias fuentes y determina si un sistema crítico o un conjunto de datos se ha visto afectado. Asesora sobre su remediación. & Análisis avanzado de forensia de redes y basado en host. Procedimientos de respuesta a incidentes, revisiones de registros, evaluación básica de malware e inteligencia de amenazas. \\ 
            \hline
            Nivel 3 - Especialista en la materia & Se trata de un conjunto de especialistas que cubren distintas áreas de un CSIRT. 
            Actúan como “cazadores” de amenazas, sin esperar que se intensifiquen los incidentes. Se encuentra estrechamente involucrado en el desarrollo, ajuste e implementación de análisis de detección de amenazas.
             & Entrenamiento avanzado en detección de anomalías. Entrenamiento específico en herramientas para la agregación y análisis de datos e inteligencia de amenazas. 
            Poseen un conocimiento profundo en áreas como redes, puntos finales, inteligencia de amenazas, forensia e ingeniería inversa de malware, así como la infraestructura de IT subyacente.
            \\ 
             \hline
            Director del CSIRT & Administra recursos para incluir personal, presupuesto, programación de turnos y estrategias para cumplir con los acuerdos de nivel de servicio. Se comunica con la gerencia y sirve como persona de contacto en el caso de incidentes críticos. Proporciona una dirección general para el CSIRT. & Gestión de proyectos, formación en gestión de respuesta a incidentes, habilidades generales de gestión de personas y comunicación institucional.  \\
            \hline %linea final de tabla
        \end{tabular}
        \caption{Integrantes de un CSIRT y sus funciones}
        \label{table:1}
    \end{table}
    
   \end{section}
   
   \begin{section}{Procesos}  
   
   \end{section}
   \begin{section}{Tecnología}  
        \begin{subsection}{Agregando contexto a los incidentes}
        La incorporación de inteligencia de amenazas y otras informaciones de contexto tales como activos e identidades, contribuye al proceso de investigación del analista de un CSIRT. En determinados casos, la información inicial que está asociada a una alerta puede ser muy limitada, por ejemplo la dirección IP del punto final sospechoso es insuficiente por sí sola para tomar una decisión. \par
        Para que los analistas puedan investigar un incidente, generalmente necesitan más información, por ejemplo los nombres del dueño y de dominio de la máquina, registros DHCP para mapear la IP con el host al momento del incidente, etc. Si el sistema de monitoreo incorpora información de identidad y de los activos de información, entre otros datos de contexto, le permitirá al analista ahorrar tiempo y esfuerzo para priorizar los incidentes y elaborar la respuesta más apropiada.\par

        \end{subsection}
        \begin{subsection}{Agregando contexto a los incidentes}
        \end{subsection}
        
        \begin{subsection}{Definición de conductas normales}
        \end{subsection}
        
        \begin{subsection}{Inteligencia de amenazas}
        \end{subsection}
        
        \begin{subsection}{Obstáculos para el manejo eficiente de incidentes del CSIRT}
        \end{subsection}
   \end{section}
      
   \begin{section}{Ámbitos de actuación de los CSIRT}
        \begin{subsection}{Estado de la ciberseguridad en Argentina} 
            \begin{subsubsection}{Demanda de ciberseguridad Argentina}
            \end{subsubsection}
        \end{subsection}
   \end{section}

    \begin{section}{SIEM: Definición y funciones}
    
    \end{section}
    \begin{section}{Soluciones disponibles}
    
    
        \begin{subsection}{Soluciones comerciales}
        
        \end{subsection}
        \begin{subsection}{Soluciones gratuitas y de código abierto}
        
        
            \begin{subsubsection}{AlienVault OSSIM}
            
            \end{subsubsection}
            \begin{subsubsection}{Graylog}
            
            \end{subsubsection}
            \begin{subsubsection}{Elastic Stack}
            
            \end{subsubsection}
            \begin{subsubsection}{Security Onion}
            
            \end{subsubsection}
            
        \end{subsection} 
    \end{section}
    \begin{section}{Corolario}
    
    \end{section}
            
            
            
            

    \chapter{Descripción de Requerimientos}
    Con el objetivo de desplegar un sistema SIEM capaz de soportar los requerimientos funcionales y no funcionales, es necesario definir el entorno en el que operará la plataforma. \par
    Para esto se requiere, en primer lugar describir la topología de la red de la organización: realizar un relevamiento de las conexiones  de la infraestructura de red interna de la unidad central, las de sus dependencias y la red entre las unidades geográficamente distribuidas si las hubiera. Debe incluirse la topología de las conexiones de salida a Internet. \par
    %En segundo lugar será necesario inventariar los activos de la organización: se requerirá una investigación y un relevamiento de los activos con los que cuentan las infraestructuras de red y de datos a fin de clasificarlos. \par
    Las tareas de relevamiento anteriormente descritas proporcionarán un entendimiento acabado y profundo de la situación en la que se encuentra la infraestructura. Como resultado, será posible identificar puntos críticos a tener en consideración y como consecuencia, elegir la solución que mejor se ajuste a las necesidades de la organización. \par

    \begin{section}{Requerimientos funcionales del SIEM}
    \begin{enumerate}
        \item Recolectar y almacenar datos de incidentes de seguridad en la infraestructura de la red corporativa.
        \item Recolectar y almacenar información contextual y asociada a los activos vinculados  al incidente.
        \item Visualizar las alertas en un tablero de mando. 
        \item Implementar un sistema de envío de alertas de seguridad que notifique a los responsables de activos de información afectados.
        \item Definir un criterio para priorizar alertas.
        \item Implementar un sistema de correlación de alertas de seguridad.
    \end{enumerate}
        
    \end{section}
    
    \begin{section}{Requerimientos no funcionales}
    \begin{enumerate}
        \item La solución propuesta debe utilizar software libre.
        \item El sistema operativo base debe ser tipo Unix y abierto.
        \item La arquitectura de la solución debe ser escalable a demanda de la organización.
        \item Se requiere un despliegue automatizado de la solución.
    \end{enumerate}

    \end{section}
    
    \begin{section}{Análisis de riesgo}
    En primer lugar, será seleccionada una plataforma cuyo código sea libre y abierto. Posteriormente será elegido un sistema operativo libre, tipo Unix, que sea compatible con la solución escogida. Se tendrá en cuenta el desarrollo de una arquitectura de despliegue que contemple la escalabilidad horizontal de la solución para adaptarse a las necesidades de la organización. \par
    Se adaptará la solución para recolectar y almacenar los datos pertinentes a los incidentes de seguridad que ocurren en la red corporativa, así como la información de contexto de los activos de información que se ven afectados.  Luego de haber recibido y almacenado los datos, se configurará la solución para visualizar la información disponible. Esto permitirá implementar un sistema de envío de alertas que notifiquen a los responsables de los activos de información que se viesen comprometidos. \par
    Finalmente, se dispondrá de un sistema que permita correlacionar y filtrar alertas, en base a políticas a definir que incluyan las categorías de eventos y sus prioridades asociadas.\par
	%Finalmente se procederá a automatizar el proceso completo que comprende la implementación, despliegue y configuración de la solución junto a sus componentes asociados, mediante el uso de herramientas de automatización de tecnologías de la información. \par
    \end{section}
    
    
    \chapter{Iteración 1: “Elección de la solución”}
    \begin{section}{Security Onion como sistema de gestión de eventos}
    \begin{figure}[H]
        \centering
        \includegraphics[width=0.7\textwidth]{./iteracion_1_imagenes/figura_15_logo_sonion.png}
        \caption{logo de Security Onion\cite{sonion}}
        \label{fig:logo_sonion}
    \end{figure}
        La elección de Security Onion como plataforma se justificó en su naturaleza de código abierto y por sus características destacables respecto de otras soluciones libres, como el soporte de una activa comunidad, el desarrollo continuo de mejoras, actualizaciones y correcciones, su capacidad polimórfica y funcional de actuar como IDS, plataforma SIEM o cluster de almacenamiento. Esto permitió desarrollar distintas arquitecturas de una manera fácil y asistida para el despliegue y la consiguiente optimización de los recursos de hardware y de red. \par
        Otras de las propiedades destacables es la capacidad de integración directa con un conjunto casi universal de los sistemas IDS disponibles, tanto libres como propietarios. Security Onion también incluye un paquete de configuraciones iniciales predefinidas para la infraestructura inicial del sistema, tales como el almacenamiento, normalización y gestión de logs (pila Elastic),  los sistemas IDS y de gestión de usuarios, entre otros. \par
    \end{section}
    \begin{section}{Arquitectura del sistema de gestión de eventos}
    \end{section}
    \begin{subsection}{Arquitectura de alto nivel}
    \begin{figure}[H]
        \centering
        \includegraphics[width=0.7\textwidth]{./iteracion_1_imagenes/figura_16_arq_alto_nivel_sonion.png}
        \caption{Security Onion\cite{sonion}: Arquitectura de alto nivel}
        \label{fig:arq_top_sonion}
    \end{figure}
    En la Figura \ref{fig:arq_top_sonion} se observa la distribución de Security Onion y el flujo de datos entre sus componentes principales (la pila Elastic) y secundarios (Curator \cite{curator}, ElastAlert\cite{elasalert}, freqServer\cite{freqserver} y domainStats\cite{domainstat}). Se puede apreciar la conexión con los sistemas de detección IDS como Bro\cite{zeek}, Snort\cite{snort}, Suricata\cite{suricata}, Syslog, etc. Se distinguen también los enlaces con los puntos de administración de los analistas del CSIRT y con los servicios web externos para el envío y recepción de alertas, notificaciones, análisis de tráfico, entre otros. Un punto a destacar es que la pila Elastic se encuentra desplegada en contenedores Docker\cite{docker}. 
    \end{subsection}
    \pagebreak
    \begin{subsubsection}{Tipo de Nodos}
        \begin{itemize}
          \item Nodo Master: este nodo ejecuta su propia copia de la base de datos Elasticsearch, con la que gestiona las búsquedas a través del cluster y estructura a otros nodos en el momento de su despliegue. Lo anterior implica que puede realizar las configuraciones necesarias para los nodos de los tipos “densos” y los de almacenamiento, pero no los de sensores o Forward, por carecer estos últimos de una pila Elastic. Este nodo permite a un analista conectarse mediante un enlace de supervisión para realizar consultas de los datos.
          \begin{itemize}
              \item  Este nodo contiene los siguientes componentes:
              \begin{itemize}
                  \item Elasticsearch \cite{elastic}
                  \item Logstash \cite{elastic}
                  \item Kibana \cite{elastic}
                  \item Curator \cite{curator}
                  \item ElastAlert \cite{elasalert}
                  \item Redis \cite{redis}
                  \item Wazuh \cite{wazuh} / OSSEC \cite{ossec}
                  \item Sguild \cite{sguil}
              \end{itemize}
          \end{itemize}
          Elasticsearch \cite{elastic}, Kibana \cite{elastic} y Logstash \cite{elastic} son componentes de la pila Elastic, que se tratarán en la siguiente sección junto a ElastAlert \cite{elasalert}. El objetivo de Curator \cite{curator} y Redis \cite{redis} es administrar y optimizar las bases de datos de los nodos de almacenamiento; Wazuh \cite{wazuh} es un IDS y Security Onion lo utiliza para el monitoreo de sí mismo, configurando un sistema HIDS ad hoc, aunque es posible desplegarlo en otros nodos o puntos de interés. Sguild \cite{sguil} permite consultar eventos de una base de datos MySQL desde dentro de Security Onion y muestra los resultados en una GUI. Además, actúa como intermediario de otros componentes secundarios como Squert \cite{squert}, del que detallaremos sus funciones y comportamiento en una sección posterior. 
          \item Nodos Forward: este nodo cumple la función de procesar el tráfico y reenviar los resultados al nodo master. Los logs generados por Snort / Suricata y Bro son enviados mediante syslog a Logstash en el nodo master, utilizando un túnel ssh, donde finalmente son guardados en la base de datos Elasticsearch, donde pueden ser reenviados a los nodos de almacenamiento. Los logs pueden ser consultados a través de una búsqueda en el cluster
            \begin{itemize}
               \item Los componentes de un nodo Forward son:
               \begin{itemize}
                   \item Zeek \cite{zeek} (sucesor de Bro)
                   \item Snort \cite{snort} / Suricata \cite{suricata}
                   \item Netsniff-ng \cite{netsniff-ng}
                   \item Wazuh \cite{wazuh} / OSSEC \cite{ossec}
                   \item Syslog-ng \cite{syslog-ng}
               \end{itemize}
            \end{itemize}
            Zeek, Snort / Suricata y Netsniff-ng son procesadores de tráfico (IDS), donde Snort y Suricata serán tratados en una sección posterior. Syslog-ng es utilizado para recolectar logs de los IDS y enviarlos al Logstash del master, donde serán procesados y tratados antes de ser escritos en Elasticsearch.
            \item Nodos Pesados: Es un nodo híbrido entre el nodo Forward y el nodo Master, que incluye todos los componentes del nodo Forward, además de una instancia completa de la pila Elastic. Los nodos pesados envían los resultados de las consultas de su instancia local de Elasticsearch a las solicitudes realizadas por el nodo master mediante un túnel de autossh.
            \begin{itemize}
                \item Los componentes de este nodo son:
                \begin{itemize}
                    \item Elasticsearch
                    \item Logstash
                    \item Curator
                    \item Zeek
                    \item Snort / Suricata
                    \item Netsniff-ng
                    \item Wazuh / OSSEC
                    \item Syslog-ng (envía los logs a la instancia local de Logstash)
                \end{itemize}
            \end{itemize}
            \item Nodos de almacenamiento: su objetivo es extender las capacidades de almacenamiento y procesamiento del nodo master. Estos nodos despliegan una instancia local de la pila Elastic. De manera análoga a los nodos pesados, cuando se realiza una consulta por parte de la instancia Elasticsearch del nodo master, esta es procesada por la instancia local del nodo de almacenamiento y devuelta por un túnel autossh.
            \begin{itemize}
                \item Los componentes del nodo de almacenamiento son:
                \begin{itemize}
                    \item Elasticsearch
                    \item Logstash
                    \item Curator
                    \item Wazuh / OSSEC
                \end{itemize}
            \end{itemize}
        \end{itemize}
    \end{subsubsection}
    \pagebreak
    \begin{subsubsection}{Tipos de Arquitectura}
      La versatilidad de disponer de múltiples arquitecturas permite adaptar la plataforma a las necesidades de la organización en la que se implante. A continuación, se describen cada una de las opciones posibles:
    \begin{itemize}
         \item Arquitectura monolítica: Consiste en un único servidor que ejecuta simultáneamente los componentes centrales o propios de un nodo master y los de un nodo sensor. Es un modo híbrido y concentrado que no se recomienda para enlaces de red de alto rendimiento por los elevados requerimientos de hardware necesarios. 
         Este tipo de arquitectura se recomienda para propósitos de pruebas en laboratorio y en entornos de baja demanda de tráfico de red.
        \begin{figure}[H]
            \centering
            \includegraphics[width=0.7\textwidth]{./iteracion_1_imagenes/figura_17_arq_monolitica_sonion.png}
            \caption{Arquitectura monolítica de Security Onion\cite{sonion}}
            \label{fig:arq_monolitica_sonion}
        \end{figure}
        \FloatBarrier
        \pagebreak
        \item Arquitectura densamente distribuida: consiste en uno o más nodos pesados conectados a un nodo master. Solo se recomienda en el caso de que no sea posible desplegar una arquitectura distribuida, ya que tiene las mismas deficiencias de rendimiento de la arquitectura monolítica y no es apropiado para entornos de producción y/o enlaces de red de alta velocidad.
        \begin{figure}[H]
            \centering
            \includegraphics[width=0.7\textwidth]{./iteracion_1_imagenes/figura_18_arq_densa_sonion.png}
            \caption{Arquitectura densamente distribuida de Security Onion\cite{sonion}}
            \label{fig:arq_densa_sonion}
        \end{figure}
        \FloatBarrier
        \pagebreak
        \item Arquitectura Distribuida: consiste en un servidor master, uno o más nodos Forward y uno o más nodos de almacenamiento. Es el tipo de despliegue recomendado en términos de eficiencia de requerimientos de hardware, balance de la carga y almacenamiento de datos y optimización general de los recursos disponibles en la organización. 
        \begin{figure}[H]
            \centering
            \includegraphics[width=0.7\textwidth]{./iteracion_1_imagenes/figura_19_arq_distribuida_sonion.png}
            \caption{Arquitectura distribuida de Security Onion\cite{sonion}}
            \label{fig:arq_distribuida_sonion}
        \end{figure}
     \end{itemize}
   \end{subsubsection}
   \pagebreak
   
   \begin{section}{Arquitectura del despliegue }
    \begin{figure}[H]
        \centering
        \includegraphics[width=1\textwidth]{./iteracion_1_imagenes/figura_33_arquitectura_despliegue_proyecto.png}
        \caption{Arquitectura de Despliegue}
        \label{fig:arquitectura_despliegue_proyecto}
    \end{figure}
    \FloatBarrier
    En la Figura \ref{fig:arquitectura_despliegue_proyecto} se muestra la arquitectura de despliegue del proyecto. La descripción, de izquierda a derecha, es: el proveedor ISP de conexión a internet y por consiguiente al exterior de la organización, el switch de capa 3 al que están conectadas las dependencias cuyos enlaces fueron seleccionados para ser monitoreados para este proyecto, los nodos Forward de Security Onion y un switch de la red interna del CSIRT. Se observa que los enlaces “Dependencia 1 - switch capa 3” y el de “switch capa 3 - nodo Forward de Security Onion Dependencia 1” tienen el mismo color; esto se debe a motivos de representar el hecho de que el switch capa 3 fue configurado para reenviar el tráfico entre el enlace de este y la dependencia 1 hacia el nodo Forward mencionado. Una situación análoga ocurre entre la Dependencia 2 y el nodo Security Onion Forward Dependencia 2. \par
    El último eslabón de la conexión, el switch de capa 2, es el encargado de la red interna del CSIRT. A él se encuentran conectados las computadoras de los analistas y el nodo Master de Security Onion, los nodos Forward anteriormente mencionados y el servidor que aloja a TheHive y Cortex. Finalmente, los analistas pueden consultar y administrar los servidores correspondientes a los nodos Master y Forward de Security Onion así como al servidor que contiene a TheHive y Cortex. \par

   \end{section}
    \chapter{Iteración 2: “Configuración y despliegue en un ambiente de prueba”}
    \section{Elección del sistema base de gestión de eventos}
        \subsection{Security Onion}
    \section{Arquitectura del sistema de gestión de eventos}
        \subsection{Elastic, ElastAlert, Cortex y TheHive}
        \subsection{Recibiendo, procesando y visualizando eventos: La pila Elastic}
        \subsection{Analizando y clasificando eventos: ElastAlert}
        \subsection{El panel de control general: TheHive y Cortex}
        \subsection{Tomando acciones: TheHiveHooks}
    \section{Integración con los sistemas de detección}
        \subsection{Suricata, Snort y Ossec}
    \section{Arquitectura del despliegue en la organización}
    \chapter{Iteración 3: “Reportes de incidentes y acciones automáticas”}
    \section{Configuración del ambiente de prueba}
        \subsection{Configuración del entorno de virtualización}
        \subsection{Definición y configuración de las redes a observar}
    \subsection{Instalación de las máquinas virtuales}
        \section{Configuración inicial del sistema base}
        \subsection{Instalación y configuración de Security Onion}
    \subsection{Pruebas de configuración iniciales}
    \section{Automatización de acciones}
    \chapter{Iteración 4: “Testing del sistema”}
    \section{Análisis de prioridades de los incidentes}
    \section{Incidentes a reportar}
        \subsection{Dependencias, activos y administradores de redes}
    \section{Automatización de acciones de acuerdo al incidente}
    \chapter{Iteración 5: “Testing del sistema”}
    \begin{section}{Test de detección de ataques varios}
    \end{section}
    \begin{section}{Test de reportes de incidentes}
    \end{section}
   
    \chapter{Conclusión}
El empleo y consumo masivo de las tecnologías de la información, así como la convergencia e interconexión de redes y sistemas, ha generado nuevos tipos de riesgos y amenazas para las organizaciones. Los ataques han evolucionado en complejidad, sigilo y especialización de los objetivos, requiriendo mayores esfuerzos para la prevención, detección y mitigación de los incidentes. Esto produce que las organizaciones tengan necesidad de desplegar soluciones del tipo CSIRT. \par
El SIEM, como núcleo de las operaciones de un CSIRT, tiene una gran importancia en cuanto a la recolección de datos, su análisis y las decisiones tomadas en consecuencia. \par
La solución elegida, Security Onion, representa una excelente alternativa a las soluciones comerciales ya que sus capacidades permiten cumplir los objetivos de cualquier organización. Por su naturaleza de código abierto, puede ser adaptado a configuraciones muy específicas y su desarrollo continuo, lo que nos permitio configurar fácilmente el despliegue inicial, una integración sencilla con los sensores, una presentación intuitiva de las alertas e información contextual. Por otro lado, es remarcable su grado de integración con soluciones complementarias como TheHive y Cortex. Se destaca también la documentación, que por su grado de detalle, facilitó el desarrollo de nuestro proyecto. \par
Una de las características más sobresalientes de Security Onion es que se trata de un sistema operativo en sí mismo, con distintos grados de modularización que hacen posible desarrollar distintos tipos de arquitecturas según la situación requerida. Esto lo diferencia de otras plataformas y soluciones evaluadas que consisten en un software que necesita un sistema operativo base sobre el cual desplegarse, lo que condiciona su flexibilidad para abordar distintos requerimientos o introduce limitaciones al proyecto. \par
Finalmente nos agrado saber que existen herramientas libres y gratuitas que, si bien requieren un esfuerzo para implementarlas, están a la altura de las soluciones propietarias. Si bien a nivel global la mayoría de las organizaciones optan por soluciones comerciales, existen alternativas libres que permiten desarrollar centros de monitoreo a la medida de las necesidades de una organización sin depender de terceros.

    \chapter*{Futuros trabajos}
\addcontentsline{toc}{chapter}{Futuros trabajos}
    
    % Bibliografía 
    \printbibliography
    \input{11_Anexos.tex}
    

\end{document}
