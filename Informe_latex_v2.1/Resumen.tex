\chapter*{\Large Resumen}
\addcontentsline{toc}{chapter}{Resumen}

Los servicios digitales de procesamiento y almacenamiento de información se transformaron en la columna vertebral de organizaciones de todo tipo, ya que estas volcaron masivamente su actividad a Internet en la última década. Esto genera un sinnúmero de oportunidades para eventuales atacantes, que buscan detectar y explotar las vulnerabilidades de los sistemas en los cuales se apoya la infraestructura de las organizaciones. \par
El proceso de monitoreo de la seguridad de una red de datos compleja requiere recopilar diferentes tipos de datos para detectar, verificar y contener acciones ofensivas. Es necesario contar con un sistema SIEM que centralice la información proveniente de múltiples fuentes, distribuidas a lo largo de toda la red de datos. En este trabajo se propuso, implementó y desplegó una solución SIEM en la Universidad Nacional de Córdoba.\par
















