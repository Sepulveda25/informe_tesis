\chapter{Conclusión}
El empleo y consumo masivo de las tecnologías de la información, así como la convergencia e interconexión de redes y sistemas, ha generado nuevos tipos de riesgos y amenazas para las organizaciones. Los ataques han evolucionado en complejidad, sigilo y especialización de los objetivos, requiriendo mayores esfuerzos para la prevención, detección y mitigación de los incidentes. Esto produce que las organizaciones tengan necesidad de desplegar soluciones del tipo CSIRT. \par
El SIEM, como núcleo de las operaciones de un CSIRT, tiene una gran importancia en cuanto a la recolección de datos, su análisis y las decisiones tomadas en consecuencia. \par
La solución elegida, Security Onion, representa una excelente alternativa a las soluciones comerciales ya que sus capacidades permiten cumplir los objetivos de cualquier organización. Por su naturaleza de código abierto, puede ser adaptado a configuraciones muy específicas y su desarrollo continuo, lo que nos permitio configurar fácilmente el despliegue inicial, una integración sencilla con los sensores, una presentación intuitiva de las alertas e información contextual. Por otro lado, es remarcable su grado de integración con soluciones complementarias como TheHive y Cortex. Se destaca también la documentación, que por su grado de detalle, facilitó el desarrollo de nuestro proyecto. \par
Una de las características más sobresalientes de Security Onion es que se trata de un sistema operativo en sí mismo, con distintos grados de modularización que hacen posible desarrollar distintos tipos de arquitecturas según la situación requerida. Esto lo diferencia de otras plataformas y soluciones evaluadas que consisten en un software que necesita un sistema operativo base sobre el cual desplegarse, lo que condiciona su flexibilidad para abordar distintos requerimientos o introduce limitaciones al proyecto. \par
Finalmente nos agrado saber que existen herramientas libres y gratuitas que, si bien requieren un esfuerzo para implementarlas, están a la altura de las soluciones propietarias. Si bien a nivel global la mayoría de las organizaciones optan por soluciones comerciales, existen alternativas libres que permiten desarrollar centros de monitoreo a la medida de las necesidades de una organización sin depender de terceros.
